% Options for packages loaded elsewhere
\PassOptionsToPackage{unicode}{hyperref}
\PassOptionsToPackage{hyphens}{url}
\PassOptionsToPackage{dvipsnames,svgnames,x11names}{xcolor}
%
\documentclass[
]{article}
\usepackage{amsmath,amssymb}
\usepackage{setspace}
\usepackage{iftex}
\ifPDFTeX
  \usepackage[T1]{fontenc}
  \usepackage[utf8]{inputenc}
  \usepackage{textcomp} % provide euro and other symbols
\else % if luatex or xetex
  \usepackage{unicode-math} % this also loads fontspec
  \defaultfontfeatures{Scale=MatchLowercase}
  \defaultfontfeatures[\rmfamily]{Ligatures=TeX,Scale=1}
\fi
\usepackage{lmodern}
\ifPDFTeX\else
  % xetex/luatex font selection
\fi
% Use upquote if available, for straight quotes in verbatim environments
\IfFileExists{upquote.sty}{\usepackage{upquote}}{}
\IfFileExists{microtype.sty}{% use microtype if available
  \usepackage[]{microtype}
  \UseMicrotypeSet[protrusion]{basicmath} % disable protrusion for tt fonts
}{}
\makeatletter
\@ifundefined{KOMAClassName}{% if non-KOMA class
  \IfFileExists{parskip.sty}{%
    \usepackage{parskip}
  }{% else
    \setlength{\parindent}{0pt}
    \setlength{\parskip}{6pt plus 2pt minus 1pt}}
}{% if KOMA class
  \KOMAoptions{parskip=half}}
\makeatother
\usepackage{xcolor}
\usepackage[margin=1in]{geometry}
\usepackage{graphicx}
\makeatletter
\def\maxwidth{\ifdim\Gin@nat@width>\linewidth\linewidth\else\Gin@nat@width\fi}
\def\maxheight{\ifdim\Gin@nat@height>\textheight\textheight\else\Gin@nat@height\fi}
\makeatother
% Scale images if necessary, so that they will not overflow the page
% margins by default, and it is still possible to overwrite the defaults
% using explicit options in \includegraphics[width, height, ...]{}
\setkeys{Gin}{width=\maxwidth,height=\maxheight,keepaspectratio}
% Set default figure placement to htbp
\makeatletter
\def\fps@figure{htbp}
\makeatother
\setlength{\emergencystretch}{3em} % prevent overfull lines
\providecommand{\tightlist}{%
  \setlength{\itemsep}{0pt}\setlength{\parskip}{0pt}}
\setcounter{secnumdepth}{-\maxdimen} % remove section numbering
% definitions for citeproc citations
\NewDocumentCommand\citeproctext{}{}
\NewDocumentCommand\citeproc{mm}{%
  \begingroup\def\citeproctext{#2}\cite{#1}\endgroup}
\makeatletter
 % allow citations to break across lines
 \let\@cite@ofmt\@firstofone
 % avoid brackets around text for \cite:
 \def\@biblabel#1{}
 \def\@cite#1#2{{#1\if@tempswa , #2\fi}}
\makeatother
\newlength{\cslhangindent}
\setlength{\cslhangindent}{1.5em}
\newlength{\csllabelwidth}
\setlength{\csllabelwidth}{3em}
\newenvironment{CSLReferences}[2] % #1 hanging-indent, #2 entry-spacing
 {\begin{list}{}{%
  \setlength{\itemindent}{0pt}
  \setlength{\leftmargin}{0pt}
  \setlength{\parsep}{0pt}
  % turn on hanging indent if param 1 is 1
  \ifodd #1
   \setlength{\leftmargin}{\cslhangindent}
   \setlength{\itemindent}{-1\cslhangindent}
  \fi
  % set entry spacing
  \setlength{\itemsep}{#2\baselineskip}}}
 {\end{list}}
\usepackage{calc}
\newcommand{\CSLBlock}[1]{\hfill\break\parbox[t]{\linewidth}{\strut\ignorespaces#1\strut}}
\newcommand{\CSLLeftMargin}[1]{\parbox[t]{\csllabelwidth}{\strut#1\strut}}
\newcommand{\CSLRightInline}[1]{\parbox[t]{\linewidth - \csllabelwidth}{\strut#1\strut}}
\newcommand{\CSLIndent}[1]{\hspace{\cslhangindent}#1}
\usepackage{fancyhdr}
\usepackage{lineno}
\linenumbers
\modulolinenumbers[2]
\ifLuaTeX
  \usepackage{selnolig}  % disable illegal ligatures
\fi
\usepackage{bookmark}
\IfFileExists{xurl.sty}{\usepackage{xurl}}{} % add URL line breaks if available
\urlstyle{same}
\hypersetup{
  pdftitle={Foundations of ecological and evolutionary change},
  colorlinks=true,
  linkcolor={blue},
  filecolor={Maroon},
  citecolor={Blue},
  urlcolor={Blue},
  pdfcreator={LaTeX via pandoc}}

\title{Foundations of ecological and evolutionary change}
\author{A. Bradley Duthie\(^{1,a,*}\) and Victor J. Luque\(^{2,a}\)}
\date{{[}1{]} Department of Biological and Environmental Sciences,
University of Stirling, Stirling, Scotland {[}2{]} Department of
Philosophy, University of Valencia, Valencia, Spain {[}*{]}
Corresponding author:
\href{mailto:alexander.duthie@stir.ac.uk}{\nolinkurl{alexander.duthie@stir.ac.uk}}
{[}a{]} Equal contribution}

\begin{document}
\maketitle

\setstretch{2}
\section{Abstract}\label{abstract}

Biological evolution is realised through the same mechanisms of birth
and death that underlie change in population density. The deep
interdependence between ecology and evolution is well-established, but
much theory in each discipline has been developed in isolation. While
recent work has accomplished eco-evolutionary synthesis, a gap remains
between the logical foundations of ecology and evolution. We bridge this
gap with a new equation that defines a summed value for a characteristic
across individuals in a population, from which the fundamental equations
of population ecology and evolutionary biology (the Price equation) are
derived. We thereby unify the fundamental equations of population
ecology and biological evolution under a general framework. Our
unification further demonstrates the equivalence between mean population
growth rate and evolutionary fitness and links this change to ecosystem
function. Finally, we outline how our proposed framework can be used to
unify social evolution and density-dependent population growth.

\textbf{Key words:} Ecology, Evolution, Eco-Evolutionary Theory,
Fundamental Theorem, Price Equation, Population Growth

\section{Introduction}\label{introduction}

Theoretical unification of phenomena is a powerful tool for scientific
advancement. Such unification has been a major goal in scientific
research throughout history (\citeproc{ref-Kitcher1993}{Kitcher 1993}),
and its value is perhaps most evident in reconciling unconnected models
and revealing new and unexpected empirical predictions. In evolutionary
biology, the Price equation (Box 1) provides a unifying framework for
evolutionary theory by exhaustively and exactly describing evolutionary
change for any closed evolving population
(\citeproc{ref-Price1970}{Price 1970}; \citeproc{ref-Luque2016}{Luque
2017}; \citeproc{ref-Lehtonen2020}{Lehtonen et al. 2020}). The Price
equation is therefore fundamental, in the sense that it binds together
all of evolutionary theory by formally defining what evolutionary change
is and is not (\citeproc{ref-Price1970}{Price 1970};
\citeproc{ref-Rice2004}{Rice 2004}; \citeproc{ref-Gardner2008}{Gardner
2008}; \citeproc{ref-Frank2015a}{Frank 2017};
\citeproc{ref-Luque2016}{Luque 2017}; \citeproc{ref-Luque2021}{Luque and
Baravalle 2021}). Using this formal definition, the scope of, and
relationships among, sub-disciplines within evolutionary theory can be
clarified. For example, fundamental equations of both population and
quantitative genetics can be derived from the Price equation
(\citeproc{ref-Queller2017}{Queller 2017}). This provides conceptual
clarity by demonstrating the logical consistency of different
theoretical frameworks within evolutionary biology. Our aim here is to
propose an equation that extends this conceptual clarity to include
population change and thereby provide a formal and exact definition of
joint ecological and evolutionary change.

In biological populations, ecological change is caused by the same
processes of individual birth and death that cause changes in allele
frequencies and phenotypes (\citeproc{ref-Turchin2001}{Turchin 2001};
\citeproc{ref-Connor2004}{Connor and Hartl 2004};
\citeproc{ref-Barfield2011}{Barfield et al. 2011}). As with evolution, a
fundamental equation can exhaustively and exactly define population
change. Unlike the Price equation, this fundamental equation is perhaps
self-evident; population change is simply the addition of births and
subtraction of deaths from current population size (\(N_{t}\)) to
recover the new population size (\(N_{t+1}\); Box 2). By definition,
this relationship of \(N_{t+1} = N_{t} + Births - Deaths\) applies to
any closed population. Turchin (\citeproc{ref-Turchin2001}{2001}) argues
that general principles are needed to establish a logical foundation for
population ecology, and this simple birth and death model and the
consequences that logically follow from it (e.g., exponential population
growth) is fundamental to population ecology. Any unifying definition of
joint ecological and evolutionary change must be able to derive both the
Price equation and this birth and death model.

The interdependence of ecological and evolutionary processes has long
been recognised (e.g., \citeproc{ref-Darwin1859}{Darwin 1859};
\citeproc{ref-Fisher1958}{Fisher 1958};
\citeproc{ref-Pelletier2009}{Pelletier et al. 2009}), but the rise of
eco-evolutionary models, which incorporate both, is relatively recent
following a widespread recognition that ecology and evolution can happen
on similar timescales (\citeproc{ref-Govaert2019}{Govaert et al. 2019};
\citeproc{ref-Yamamichi2023}{Yamamichi et al. 2023}). Currently, a
universally recognised formal definition of eco-evolutionary change is
lacking, with some authors broadly interpreting ``eco-evolutionary
dynamics'' to allow for a separation of ecological and evolutionary
timescales (\citeproc{ref-Lion2023}{Lion et al. 2023}) and others
advocating for a more narrow interpretation in which no such separation
is permitted (\citeproc{ref-Bassar2021}{Bassar et al. 2021}). Bassar et
al. (\citeproc{ref-Bassar2021}{2021}) recognise two types of
eco-evolutionary models that follow from these interpretations. The
first type uses separate equations to model population change versus
evolutionary change, thereby allowing for any number of ecological,
evolutionary, or environmental feedbacks (e.g.,
\citeproc{ref-Lion2018}{Lion 2018}; \citeproc{ref-Patel2018}{Patel et
al. 2018}; \citeproc{ref-Lion2023}{Lion et al. 2023}). The second type
models population demographics as functions of quantitative traits, with
ecological and evolutionary change following from demographic processes
and trait distributions operating on the same timescale (e.g.,
\citeproc{ref-Barfield2011}{Barfield et al. 2011};
\citeproc{ref-Simmonds2020}{Simmonds et al. 2020};
\citeproc{ref-Jaggi2024}{Jaggi et al. 2024}). Both model types can be
very general, but like all predictive models, they rely on simplifying
assumptions for tractability (\citeproc{ref-Levins1966}{Levins 1966};
\citeproc{ref-Luque2016}{Luque 2017}). These simplifying assumptions can
be grounded in the Price equation to demonstrate accuracy and logical
consistency when modelling evolutionary change (e.g.,
\citeproc{ref-Coulson2008}{Coulson and Tuljapurkar 2008};
\citeproc{ref-Barfield2011}{Barfield et al. 2011};
\citeproc{ref-Rees2016}{Rees and Ellner 2016};
\citeproc{ref-Lion2018}{Lion 2018}). For example, Barfield et al.
(\citeproc{ref-Barfield2011}{2011}) link their model back to Price
(\citeproc{ref-Price1970}{1970}), which they consider to be a
``universal law of evolution'', to place their conclusions concerning
stage structured evolution in the broader context of evolutionary
theory. The role of fundamental equations is therefore important for
unifying theory (\citeproc{ref-Luque2016}{Luque 2017}), and we believe
that a fundamental equation of eco-evolutionary change is needed.

We present an equation from which the fundamental equations of ecology
and evolutionary biology can be derived. Derivation follows by adding
assumptions that are specific to population ecology or evolution in the
same way that key equations of population genetics (e.g., average
excess) or quantitative genetics (e.g., Breeder's equation) can be
derived from the Price equation (\citeproc{ref-Queller2017}{Queller
2017}). We propose our equation as a formal definition of
eco-evolutionary change.

\begin{center}\rule{0.5\linewidth}{0.5pt}\end{center}

\begin{quote}
\textbf{Box 1:} The Price equation is an abstract formula to represent
evolutionary change. Formulated originally in the early 1970s by George
Price (\citeproc{ref-Price1970}{Price 1970},
\citeproc{ref-Price1972}{1972}), it postulates some basic properties
that all evolutionary systems must satisfy: change over time, ancestor
and descendant relations, and a character or phenotype
(\citeproc{ref-Rice2004}{Rice 2004}). Using simple algebraic language,
the Price equation represents evolutionary change with the predominant
notation,
\[\bar{w}\Delta\bar{z} = \mathrm{Cov}\left(w, z\right) + \mathrm{E}\left(w\Delta z\right).\]
In the above equation, \(\Delta\bar{z}\) is the change in the average
character value \(z\) over a time step of arbitrary length, \(w\) is an
individual's absolute fitness, and \(\bar{w}\) average population
fitness. On the right-hand side of the equation, the first term is the
covariance between a character value \(z\) and fitness \(w\), which
reflects \(\bar{z}\) change attributable to differential survival and
reproduction. The second term is the expected value of \(\Delta z\),
which reflects the extent to which offspring deviate from parents in
\(z\) (\citeproc{ref-Rice2004}{Rice 2004};
\citeproc{ref-Okasha2006}{Okasha 2006}; \citeproc{ref-Frank2012}{Frank
2012}). A more specific version of the covariance term was already known
within the quantitative and population genetics tradition
(\citeproc{ref-Robertson1966}{Robertson 1966}), usually representing the
action of natural selection. The Price equation adds an expectation term
and abstracts away from any specific mechanisms of replication or
reproduction, or mechanisms of inheritance. Its definitional nature and
lack of substantive biological assumptions has been portrayed both as a
strength (\citeproc{ref-Rice2004}{Rice 2004};
\citeproc{ref-Frank2012}{Frank 2012}; \citeproc{ref-Luque2016}{Luque
2017}; \citeproc{ref-Baravalle2022}{Baravalle and Luque 2022}), and its
greatest weakness. The abstract nature of the Price equation places it
at the top of the hierarchy of fundamental theorems of evolution from
which the rest (Robertson's theorem, Fisher's fundamental theorem,
breeder's equation, Hamilton's rule, adaptive dynamics, etc.) can be
easily derived (\citeproc{ref-Lehtonen2016}{Lehtonen 2016},
\citeproc{ref-Lehtonen2018}{2018}; \citeproc{ref-Queller2017}{Queller
2017}). This abstractness is also key to developing a more general view
of evolution (\citeproc{ref-Rice2020}{Rice 2020};
\citeproc{ref-Luque2021}{Luque and Baravalle 2021};
\citeproc{ref-Edelaar2023}{Edelaar et al. 2023}). In contrast, some
researchers consider the Price equation just a triviality (even
tautological), and useless without further modelling assumptions
(\citeproc{ref-VanVeelen2005}{van Veelen 2005};
\citeproc{ref-VanVeelen2012}{van Veelen et al. 2012}). The debate
remains open (\citeproc{ref-VanVeelen2020}{van Veelen 2020};
\citeproc{ref-Baravalle2024}{Baravalle et al. 2024}).
\end{quote}

\begin{center}\rule{0.5\linewidth}{0.5pt}\end{center}

\begin{center}\rule{0.5\linewidth}{0.5pt}\end{center}

\begin{quote}
\textbf{Box 2:} The number of individuals in any closed population
(\(N\)) at any given time (\(t + 1\)) is determined by the existing
number of individuals (\(N_{t}\)), plus the number of births
(\(Births\)) minus the number of deaths (\(Deaths\)),
\[N_{t+1} = N_{t} + Births - Deaths.\] This equation is necessarily true
for any closed population. Despite its simplicity, it is a general
equation for defining population change and a starting point for
understanding population ecology. Turchin
(\citeproc{ref-Turchin2001}{2001}) notes that a consequence of this
fundamental equation is the tendency for populations to grow
exponentially (technically geometrically in the above case where time is
discrete). This inherent underlying tendency towards exponential growth
persists even as the complexities of real populations, such as
structure, stochasticity, or density-dependent effects are added to
population models (\citeproc{ref-Turchin2001}{Turchin 2001}). Given the
assumption that all individuals in the population are identical, a per
capita rate of birth, \(Births = bN_{t}\) and death, \(Deaths = dN_{t}\)
can be defined. Rearranging and defining \(\lambda = 1 + b - d\) gives,
\(N_{t+1} = N_{t}\lambda\). Here \(\lambda\) is the finite rate of
increase (\citeproc{ref-Gotelli2001}{Gotelli 2001}), and note that
because \(0 \leq d \leq 1\), \(\lambda \geq 0\). Verbally, the change in
size of any closed population equals its existing size times its finite
rate of increase.
\end{quote}

\begin{center}\rule{0.5\linewidth}{0.5pt}\end{center}

\section{A foundation for biological evolution and population
ecology}\label{a-foundation-for-biological-evolution-and-population-ecology}

To fully unify biological evolution and population ecology, we must
reconcile the the Price equation (Box 1) with the general equation for
population change (Box 2). The Price equation is critical for
partitioning different components of biological change
(\citeproc{ref-Price1970}{Price 1970}; \citeproc{ref-Frank1997}{Frank
1997}; \citeproc{ref-Gardner2008}{Gardner 2008};
\citeproc{ref-Luque2016}{Luque 2017}; \citeproc{ref-Queller2017}{Queller
2017}; \citeproc{ref-Lehtonen2018}{Lehtonen 2018}). It has also been
highly useful for integrating evolutionary theory across disciplines
(\citeproc{ref-Fox2006}{Fox 2006};
\citeproc{ref-Brantingham2007}{Brantingham 2007};
\citeproc{ref-MacCallum2012}{MacCallum et al. 2012};
\citeproc{ref-Frank2015}{Frank 2015}; \citeproc{ref-Godsoe2021}{Godsoe
et al. 2021}; \citeproc{ref-Ulrich2024}{Ulrich et al. 2024}). These
properties would seem to make it an intuitive starting point for a
logical foundation of ecology and evolution, perhaps through some kind
of mathematical equivalence (\citeproc{ref-Page2002}{Page and Nowak
2002}) or addition of terms (\citeproc{ref-Collins2009}{Collins and
Gardner 2009}), or through the use of its recursive structure
(\citeproc{ref-Kerr2009}{Kerr and Godfrey-Smith 2009};
\citeproc{ref-Frank2012}{Frank 2012}). But despite its flexibility, the
Price equation still relies on relative frequencies, which must by
definition sum to one (\citeproc{ref-Frank2015}{Frank 2015}). This is
because the Price equation describes the average change in a population;
the frequency of entities is scaled thereby conserving total probability
(\citeproc{ref-Frank2015}{Frank 2015}, \citeproc{ref-Frank2016}{2016}).
But to recover the fundamental principle of exponential population
growth (\citeproc{ref-Turchin2001}{Turchin 2001}), this scaling must be
avoided in a fundamental equation of ecology and evolution.

We therefore begin with the most fundamental axioms underlying the
ecology and evolution of living systems (\citeproc{ref-Rice2004}{Rice
2004}; \citeproc{ref-Rice2009}{Rice and Papadopoulos 2009}). In such
systems, diversity is discontinuous and can be defined in terms of
discrete entities (\citeproc{ref-Dobzhansky1970}{Dobzhansky 1970}). Our
framework is general enough that entities can be anything discrete, but
we will focus on each entity \(i\) as an individual organism.
Individuals give rise to new individuals through birth such that
\(\beta_{i}\) is the number of births attributable to \(i\). Individuals
are removed from the population through death such that \(\delta_{i}\)
is an indicator variable that takes a value of 1 (death of \(i\)) or 0
(persistence of \(i\)). All individuals are defined by some
characteristic \(z_{i}\), and \(\Delta z_{i}\) defines any change in
\(z_{i}\) from one time step \(t\) to the next \(t + 1\). The total
number of individuals in the population is \(N\). From this foundation,
we can define \(\Omega\) to be a summed characteristic across \(N\)
entities,

\[\Omega = \sum_{i=1}^{N} \left(\beta_{i} - \delta_{i} + 1 \right)\left(z_{i} + \Delta z_{i} \right).
\tag{1}
\]

From eqn 1, we can derive the most fundamental equations of population
ecology (Box 2) and evolutionary biology (Box 1) through an appropriate
interpretation of \(z_{i}\).

\section{Population ecology}\label{population-ecology}

To recover the general equation for population ecology (Box 2), we
define \(z_{i}\) as the identity of \(i\) belonging to the population.
In other words, we set \(z_{i} = 1\) to simply indicate that \(i\) is a
member of the population. Further, we assume that individuals do not
change species by setting \(\Delta z_{i} = 0\). In this case,

\[\Omega = \sum_{i=1}^{N} \left(\beta_{i} - \delta_{i} + 1 \right).\]

Since we assume that individuals are identical, we can drop the
subscript \(i\) such that \(\beta_{i} = \beta\) and
\(\delta_{i} = \delta\). Summing from 1 to \(N\), we can rewrite the
above,

\[\Omega = N\left(1 + \beta - \delta \right).\]

We can now interpret \(\Omega\) as the population size at \(t+1\),
\(N_{t+1}\) and note that \(N\beta\), \(N\delta\), and \(N\) are the
total births, total deaths, and size of the population at \(t\),
respectively,

\[N_{t+1} = N_{t} + Births - Deaths.
\tag{2}
\]

If we define \(\lambda = 1 + \beta - \delta\) (Box 2), then we can
rewrite,

\[N_{t+1} = N_{t}\lambda.
\tag{3}
\]

We therefore recover the general equation for population ecology (eqn 2)
and the fundamental property of exponential growth in populations
(\citeproc{ref-Turchin2001}{Turchin 2001}) (eqn 3).

\section{Evolutionary biology}\label{evolutionary-biology}

Recovering the Price equation requires a few more steps. We start by
defining individual fitness,

\[w_{i} = \beta_{i} - \delta_{i} + 1.
\tag{4}
\]

In this definition, the longevity of the individual matters. An
individual that survives from \(t\) to \(t + 1\) has a higher fitness
than one that dies, even if both have the same reproductive output. With
this definition of fitness (eqn 4), we substitute,

\[\Omega = \sum_{i=1}^{N} \left(w_{i}z_{i} + w_{i}\Delta z_{i} \right).
\tag{5}
\]

We can break eqn 5 down further and multiply each side by \(1/N\),

\[\frac{1}{N}\Omega = \frac{1}{N}\sum_{i=1}^{N} \left(w_{i}z_{i} \right) + \frac{1}{N}\sum_{i=1}^{N}\left( w_{i}\Delta z_{i} \right).
\tag{6}
\]

We can rewrite the terms on the right-hand side of eqn 6 as expected
values and remove the subscripts,

\[\frac{1}{N}\Omega = \mathrm{E}\left(w z \right) + \mathrm{E}\left( w \Delta z  \right).
\tag{7}
\]

Now we must consider the total conservation of probability
(\citeproc{ref-Frank2015}{Frank 2015}, \citeproc{ref-Frank2016}{2016}).
In eqn 7, \(\Omega\) is the total sum trait values (\(z_{i}\)) across
the entire population at \(t + 1\) divided by the number of individuals
(\(N\)) in the population at \(t\). But the size of the population can
change from \(t\) to \(t + 1\). To recover mean trait change for the
Price equation (and therefore conserve total probability), we need to
account for this change in population size. The left-hand side of eqn 7
describes contributions to the sum trait value from the new population
at \(t + 1\). But we cannot treat \(\Omega/N\) as the mean of \(z\) at
\(t+1\) (\(\bar{z}'\)) because we need to weigh \(N\) by the mean
fitness of the population at \(t\) to account for any change in
population size from \(t\) to \(t+1\).

For example, if mean fitness at \(t\) is 2 (i.e., \(\bar{w} = 2\)), then
half as many individuals will have contributed to \(\Omega\) in
\(t + 1\) than would have if \(\bar{w} = 1\) (i.e., there are \(N\)
individuals at \(t\) and \(2N\) individuals at \(t + 1\)). We therefore
need to multiply the mean trait value \(\bar{z}'\) (at \(t + 1\)) by the
mean fitness \(\bar{w}\) (at \(t\)) to recover the mean contribution of
the \(N\) individuals at \(t\) to the total \(\Omega\). Consequently,

\[\Omega = N\bar{w}\bar{z}'
\tag{8}
\]

Equation 8 conserves the total probability and now clarifies \(\Omega\)
as a summed trait value, which equals expected population growth at
\(t\) times mean trait value at \(t + 1\). This is consistent with the
population ecology derivation from the previous section where
\(z_{i} = 1\) by definition, and \(\Omega = N_{t+1}\) (note
\(\lambda = \bar{w}\)). We can therefore rewrite eqn 7,

\[\bar{w}\bar{z}' = \mathrm{E}\left(w z \right) + \mathrm{E}\left( w \Delta z  \right).
\tag{9}
\]

We can rearrange eqn 9 to derive the Price equation by expressing
covariance as,
\(\mathrm{Cov}(X,Y) = \mathrm{E}(XY) - \mathrm{E}(X)\mathrm{E}(Y)\), and
therefore
\(\mathrm{E}(XY) = \mathrm{Cov}(X,Y) + \mathrm{E}(X)\mathrm{E}(Y)\).
Substituting into eqn 9,

\[\bar{w}\bar{z}' = \mathrm{Cov}\left(w ,z \right) + \bar{w}\bar{z} + \mathrm{E}\left( w \Delta z  \right).\]

From here,

\[\bar{w}\left(\bar{z}' - \bar{z}\right) = \mathrm{Cov}\left(w ,z \right) + \mathrm{E}\left( w \Delta z  \right).\]

Since \(\Delta \bar{z} = \left(\bar{z}' - \bar{z}\right)\),

\[\bar{w}\Delta \bar{z} = \mathrm{Cov}\left(w ,z \right) + \mathrm{E}\left( w \Delta z  \right).
\tag{10}
\]

From eqn 1, which describes fundamental birth and death processes in a
population, we can derive both the most fundamental model of population
ecology (eqn 2; Box 2) and the fundamental equation of evolution (eqn
10; Box 1). We might also consider \(\Omega\) to be a quantity of
ecosystem function, in that it is the summed effect of the traits
(\(z_{i}\)) of individual entities (\(i\)) in the population. For
example, \(z\) might be interpreted as biomass or photosynthetic rate.

\section{Discussion}\label{discussion}

A classical sign of scientific progress is the ability to connect
disparate theories and models to show how empirical and theoretical
models are logical (mathematical) consequences of more fundamental ones
(\citeproc{ref-Nagel1961}{Nagel 1961}). Conceptual unification therefore
has a critical role in advancing scientific theory
(\citeproc{ref-Morrison2000}{Morrison 2000}). Showing how apparently
disparate phenomena follow from the same shared principles reveals what
is most fundamental about the natural world and provides a foundation
for further scientific investigation and the construction of more
coherent and predictive models. Our eqn 1 provides a foundation for the
conceptual unification of the fundamental equations of population
ecology and evolution.

Rather than making simplifying assumptions, as is the approach for most
ecological and evolutionary models, we focus on fundamental axioms that
are universal to closed biological systems: discrete individuals, birth,
death, and change over time. We define an abstract sum (\(\Omega\)), to
which all individuals within the population contribute. From the simple
assumptions of population identity (\(z_{i} = 1\)) and invariability
(\(\Delta z_{i} = 0\)), we recover the most general equation of
population ecology (Box 2) and principle of exponential growth
(\(N_{t+1} = N_{t}\lambda\)). By defining individual fitness (\(w_{i}\))
and applying the total conservation of probability to individual
frequencies (\citeproc{ref-Frank2015}{Frank 2015},
\citeproc{ref-Frank2016}{2016}), we recover the most fundamental
equation of evolution (Box 1). We therefore deliver a general framework
to unify fundamental equations of ecological and evolutionary change.

The Price equation provides a complete and exact description of
evolution in any closed evolving system (\citeproc{ref-Price1970}{Price
1970}; \citeproc{ref-Frank2012}{Frank 2012}). It is derived by
rearranging the mathematical notation defining changes in the
frequencies and characteristics of any type of entity
(\citeproc{ref-Price1970}{Price 1970};
\citeproc{ref-Gardner2008}{Gardner 2008}; \citeproc{ref-Luque2016}{Luque
2017}) (e.g., individuals, alleles). This derivation partitions total
characteristic change into different components, making it possible to
isolate evolutionary mechanisms (e.g., selection) and levels of
biological organisation (e.g., group, individual)
(\citeproc{ref-Frank1995}{Frank 1995}, \citeproc{ref-Frank2012}{2012};
\citeproc{ref-Kerr2009}{Kerr and Godfrey-Smith 2009};
\citeproc{ref-Luque2016}{Luque 2017}; \citeproc{ref-Okasha2020}{Okasha
and Otsuka 2020}). Because of its abstract nature and lack of any
system-specific assumptions, the Price equation makes no predictions
about what will happen in any particular system
(\citeproc{ref-Gardner2020}{Gardner 2020}). Its role is not to predict,
but to formally and completely define and separate components of
evolutionary change. The same is true of the general equation for
population change (eqns 2 and 3), at least as we have used it here where
it serves to define what population growth means in ecology. This
equation formally and completely describes population change in terms of
births and deaths. In eqn 1, we therefore have a fundamental equation
from which we can derive complete ecological and evolutionary change in
any closed biological population. We anticipate that this will be useful
for eco-evolutionary theory in the same way that the Price equation is
useful for evolutionary theory: facilitating specific model development
and identifying new conceptual insights, unresolved errors, and sources
of model disagreements.

Our unification demonstrates the equivalence between the finite rate of
increase \(\lambda\) (Box 2) (\citeproc{ref-Gotelli2001}{Gotelli 2001})
and population mean evolutionary fitness \(\bar{w}\) (Box 1). The
population growth equation \(N_{t+1} = N_{t}\lambda\) could therefore be
rewritten as \(N_{t+1} = N_{t}\bar{w}\). We do not claim to be the first
to notice this equivalence between population growth and mean population
fitness. Indeed, the relationship between population growth rate and
evolutionary fitness has been proposed and applied many times before
(\citeproc{ref-Fisher1930}{Fisher 1930};
\citeproc{ref-Charlesworth1980}{Charlesworth 1980};
\citeproc{ref-Lande1982}{Lande 1982}; \citeproc{ref-Roff2008}{Roff
2008}; \citeproc{ref-Lion2018}{Lion 2018}). For example, Lande
(\citeproc{ref-Lande1982}{1982}) explicitly concludes mean absolute
fitness per unit time is \(\bar{w} = e^{r}\), where \(r = ln(\lambda)\).
We show this from first principles and clarify the relationship between
fitness and population growth. Over an arbitrary length of time, fitness
is properly defined as \(w_{i} = \beta_{i} - \delta_{i} + 1\). Over an
individual's lifetime (which, by definition, includes death), fitness is
therefore \(\beta_{i}\). Interestingly, the rate of change in ecology
and evolution are reflected in the first and second statistical moments
of fitness, respectively. Population growth rate reflects mean fitness
\(\bar{w}\), while the rate of evolutionary change reflects the variance
in fitness \(\mathrm{Var}(w)\) (i.e., Fisher's fundamental theorem)
(\citeproc{ref-Frank1997}{Frank 1997}; \citeproc{ref-Rice2004}{Rice
2004}; \citeproc{ref-Queller2017}{Queller 2017}).

Our unification may also help explain, at least partially, some of the
success of classical population genetic models. For decades, population
genetics (and to some extent quantitative genetics) has been accused of
being a reductionistic view of evolution, reducing everything to changes
in allele frequencies and abstracting away from individuals and their
environments (the ecological interactions)
(\citeproc{ref-MacColl2011}{MacColl 2011}). This has been a line of
argumentation by some defenders of the so-called Extended Synthesis
(\citeproc{ref-Pigliucci2009}{Pigliucci 2009}), especially in relation
to niche construction (\citeproc{ref-Odling-smee2003}{Odling-Smee et al.
2003}). Famously, Mayr (\citeproc{ref-Mayr1960}{1959}) characterized
population genetics as a simple input and output of genes, analogous to
``the adding of certain beans to a beanbag and the withdrawing of
others'' (also called ``beanbag genetics'').

Historical critics of population genetics could not articulate a clear
explanation for why it works so well despite all of its idealisations
and simplifications. From the Price equation, we are able to recover
classical population and quantitative genetic models
(\citeproc{ref-Queller2017}{Queller 2017}) and develop new ones
(\citeproc{ref-Rice2004}{Rice 2004}, \citeproc{ref-Rice2020}{2020};
\citeproc{ref-Luque2016}{Luque 2017}; \citeproc{ref-Lion2018}{Lion
2018}). Our eqn 1 contains ecology at its core, and we show how the
Price equation logically follows from it after accounting for absolute
population growth (eqn 8). We therefore conclude that population and
quantitative genetic equations contain ecology (no matter how hidden),
and the ecological nature of evolution is implicit in population and
quantitative genetic models.

We have focused on the dynamics of a closed population, and in doing so
leave ecological and evolutionary change attributable to migration for
future work. In population ecology, immigration and emigration can be
incorporated by adding a term for each to the right-hand side of the
equation in Box 2 (\citeproc{ref-Gotelli2001}{Gotelli 2001}). In
evolution, because the Price equation relies on mapping
ancestor-descendant relationships, accounting for migration is more
challenging. Kerr and Godfrey-Smith (\citeproc{ref-Kerr2009}{2009})
demonstrate how the Price equation can be extended to allow for
arbitrary links between ancestors and descendants, thereby extending the
Price equation to allow for immigration and emigration. Frank
(\citeproc{ref-Frank2012}{2012}) presents a simplified version of Kerr
and Godfrey-Smith (\citeproc{ref-Kerr2009}{2009}) that allows some
fraction of descendants to be unconnected to ancestors. In both ecology
and evolution, accounting for migration is done through the use of
additional terms on the right-hand side of the equations.

Our fundamental equation is complete and exact. It therefore implicitly
includes any effects of density dependence on population growth (see Box
2), or any social effects on evolutionary change (see Box 1). We can
make both of these effects explicit with the same partition in the
summation on the right-hand side of eqn 1. To do this, we multiply the
expression
\(\left(\beta_{i} - \delta_{i} + 1 \right)\left(z_{i} + \Delta z_{i} \right)\)
by
\(\left(1 - \sum_{j=1}^{N} a_{ij}\left(z_{j} + \Delta z_{j}\right)\right)\),
where \(j\) indexes the same individuals as \(i\), and \(a_{ij}\)
defines the effect \(j\) has on the contribution of \(i\) to \(\Omega\).
In evolutionary terms, \(a_{ij}\) defines how \(j\) modulates the
fitness of \(i\) (\(w_{i}\)). In ecological terms, \(a_{ij}\) defines
how \(j\) modulates the contribution of \(i\) to total population
growth. If we define \(\alpha\) as the summed effect of all individuals
on a focal \(i\) (i.e., \(\alpha = \sum_{j=1}^{N} a_{ij}\)), then we can
partition ecological change into density-independent and
density-dependent effects and derive a discrete time logistic growth
equation. We save these derivations for a future investigations focused
on evolutionary and community ecology.

We have shown that we can derive the fundamental equations of population
ecology and biological evolution from a single unifying equation.
Lastly, we propose our eqn 1 as a potential starting point for defining
ecosystem function and further conceptual unification between ecology,
evolution, and ecosystem function. The Price equation has previously
been used to investigate ecosystem function
(\citeproc{ref-Loreau2001}{Loreau and Hector 2001};
\citeproc{ref-Fox2006}{Fox 2006}), but not with any attempt towards
conceptual unification with evolutionary biology. For example, Fox
(\citeproc{ref-Fox2006}{2006}) applied the abstract properties of the
Price equation to partition total change in ecosystem function into
separate components attributable to species richness, species
composition, and context dependent effects. This approach provides a
framework for comparing the effects of biodiversity on ecosystem
function in empirical systems (\citeproc{ref-Fox2006}{Fox 2006};
\citeproc{ref-Winfree2015}{Winfree et al. 2015};
\citeproc{ref-Mateo-Tomas2017}{Mateo-Tomás et al. 2017}). Instead, our
eqn 1 defines \(\Omega\) as total ecosystem function contributed by a
focal population (e.g., biomass production, decomposition,
photosynthetic rate). It is therefore possible to investigate
ecological, evolutionary, and ecosystem function change from the same
shared framework.

\section{Acknowledgements}\label{acknowledgements}

This manuscript was supported by joint funding between the French
Foundation for Research on Biodiversity (FRB) Centre for the Synthesis
and Analysis of Biodiversity (CESAB) and the German Centre for
Integrative Biodiversity Research (sDiv). It was written as part of the
Unification of Modern Coexistence Theory and Price Equation (UNICOP)
project. Victor J. Luque was also supported by the Spanish Ministry of
Science and Innovation (Project: PID2021-128835NB-I00), and the
Conselleria d'Innovaci\(\'{o}\), Universitats, Ci\(\`{e}\)ncia i
Societat Digital -- Generalitat Valenciana (Project: CIGE/2023/16). We
are grateful for many conversations with S\(\'{e}\)bastien Lion, Kelsey
Lyberger, Swati Patel, and especially Lynn Govaert, whose questions
helped us clarify the relationship between population growth and
fitness. Brad Duthie would also like to thank Brent Danielson and Stan
Harpole. Victor J. Luque would also like to thank Lorenzo Baravalle, Pau
Carazo, Santiago Ginnobili, Manuel Serra, and Ariel Roff\(\'{e}\).

\section{Author Contributions}\label{author-contributions}

Both authors came up with the question idea. Following many discussions
between the authors, ABD proposed the initial equation with subsequent
exploration and development from both authors. Both authors contributed
to the writing.

\section{Competing Interests}\label{competing-interests}

The authors declare no competing interests.

\section{Data Availability}\label{data-availability}

This work does not include any data.

\section*{References}\label{references}
\addcontentsline{toc}{section}{References}

\phantomsection\label{refs}
\begin{CSLReferences}{0}{0}
\bibitem[\citeproctext]{ref-Baravalle2022}
Baravalle, L., and V. J. Luque. 2022.
\href{https://doi.org/10.1387/theoria.21940}{Towards a pricean
foundation for cultural evolutionary theory}. Theoria 37:209--231.

\bibitem[\citeproctext]{ref-Baravalle2024}
Baravalle, L., A. Roffé, V. J. Luque, and S. Ginnobili. 2024.
\href{https://doi.org/10.1007/s13752-024-00482-4}{The value of price}.
Biological Theory.

\bibitem[\citeproctext]{ref-Barfield2011}
Barfield, M., R. D. Holt, and R. Gomulkiewicz. 2011.
\href{https://doi.org/10.1086/658903}{Evolution in stage-structured
populations}. American Naturalist 177:397--409.

\bibitem[\citeproctext]{ref-Bassar2021}
Bassar, R. D., T. Coulson, J. Travis, and D. N. Reznick. 2021.
\href{https://doi.org/10.1111/ele.13712}{Towards a more precise -- and
accurate -- view of eco-evolution}. Ecology Letters 24:623--625.

\bibitem[\citeproctext]{ref-Brantingham2007}
Brantingham, P. J. 2007. \href{https://doi.org/10.2307/40035853}{{A
unified evolutionary model of archaeological style and function based on
the Price equation}}. American Antiquity 72:395--416.

\bibitem[\citeproctext]{ref-Charlesworth1980}
Charlesworth, B. 1980. Evolution in age-structured populations.
Cambridge studies in mathematical biology. Cambridge University Press,
Cambridge.

\bibitem[\citeproctext]{ref-Collins2009}
Collins, S., and A. Gardner. 2009.
\href{https://doi.org/10.1111/j.1461-0248.2009.01340.x}{{Integrating
physiological, ecological and evolutionary change: A Price equation
approach}}. Ecology Letters 12:744--757.

\bibitem[\citeproctext]{ref-Connor2004}
Connor, J., and D. L. Hartl. 2004. {A premier of ecological genetics}.
Sinauer Associates Incorporated.

\bibitem[\citeproctext]{ref-Coulson2008}
Coulson, T., and S. Tuljapurkar. 2008.
\href{https://doi.org/10.1086/591693}{The dynamics of a quantitative
trait in an age-structured population living in a variable environment}.
American Naturalist 172:599--612.

\bibitem[\citeproctext]{ref-Darwin1859}
Darwin, C. 1859. The origin of species. Penguin.

\bibitem[\citeproctext]{ref-Dobzhansky1970}
Dobzhansky, T. 1970. Genetics of the evolutionary process (Vol. 139).
Columbia University Press.

\bibitem[\citeproctext]{ref-Edelaar2023}
Edelaar, Pim, J. Otsuka, and V. J. Luque. 2023.
\href{https://doi.org/10.1111/brv.12910}{{A generalised approach to the
study and understanding of adaptive evolution}}. Biological Reviews
98:352--375.

\bibitem[\citeproctext]{ref-Fisher1930}
Fisher, R. A. 1930. The genetical theory of natural selection. Oxford
University Press, Oxford, UK.

\bibitem[\citeproctext]{ref-Fisher1958}
Fisher, R. A. 1958. The genetical theory of natural selection (2nd ed.).
Dover.

\bibitem[\citeproctext]{ref-Fox2006}
Fox, J. W. 2006.
\href{https://doi.org/10.1890/0012-9658(2006)87\%5B2687:utpetp\%5D2.0.co;2}{{Using
the price equation to partition the effects of biodiversity loss on
ecosystem function}}. Ecology 87:2687--2696.

\bibitem[\citeproctext]{ref-Frank1995}
Frank, S. A. 1995. \href{https://doi.org/10.1006/jtbi.1995.0148}{{George
Price's contributions to evolutionary genetics}}. Journal of Theoretical
Biology 175:373--388.

\bibitem[\citeproctext]{ref-Frank1997}
---------. 1997.
\href{https://doi.org/10.1111/j.1558-5646.1997.tb05096.x}{{The Price
equation, Fisher's fundamental theorem, kin selection, and causal
analysis}}. Evolution 51:1712--1729.

\bibitem[\citeproctext]{ref-Frank2012}
---------. 2012.
\href{https://doi.org/10.1111/j.1420-9101.2012.02498.x}{{Natural
selection. IV. The Price equation}}. Journal of Evolutionary Biology
25:1002--1019.

\bibitem[\citeproctext]{ref-Frank2015}
---------. 2015. \href{https://doi.org/10.3390/e17107087}{{D'Alembert's
direct and inertial forces acting on populations: The price equation and
the fundamental theorem of natural selection}}. Entropy 17:7087--7100.

\bibitem[\citeproctext]{ref-Frank2016}
---------. 2016. \href{https://doi.org/10.3390/e18050192}{{Common
probability patterns arise from simple invariances}}. Entropy 18:1--22.

\bibitem[\citeproctext]{ref-Frank2015a}
---------. 2017. \href{https://doi.org/10.1002/ece3.2922}{{Universal
expressions of population change by the Price equation: natural
selection, information, and maximum entropy production}}. Ecology and
Evolution 1--16.

\bibitem[\citeproctext]{ref-Gardner2008}
Gardner, A. 2008. \href{https://doi.org/10.1016/j.cub.2008.01.005}{{The
Price equation}}. Current Biology 18:198--202.

\bibitem[\citeproctext]{ref-Gardner2020}
---------. 2020. \href{https://doi.org/10.1098/rstb.2019.0361}{{Price's
equation made clear}}. Philosophical Transactions of the Royal Society
B: Biological Sciences 375:20190361.

\bibitem[\citeproctext]{ref-Godsoe2021}
Godsoe, W., K. E. Eisen, D. Stanton, and K. M. Sirianni. 2021.
\href{https://doi.org/10.1007/s12080-020-00478-3}{Selection and
biodiversity change}. Theoretical Ecology 14:367--379.

\bibitem[\citeproctext]{ref-Gotelli2001}
Gotelli, N. J. 2001. A primer of ecology. Sinauer associate. Inc.
Sunderland, MA.

\bibitem[\citeproctext]{ref-Govaert2019}
Govaert, L., E. A. Fronhofer, S. Lion, C. Eizaguirre, D. Bonte, M. Egas,
A. P. Hendry, et al. 2019.
\href{https://doi.org/10.1111/1365-2435.13241}{{Eco-evolutionary
feedbacks---Theoretical models and perspectives}}. Functional Ecology
33:13--30.

\bibitem[\citeproctext]{ref-Jaggi2024}
Jaggi, H., W. Zuo, R. Kentie, J. M. Gaillard, T. Coulson, and S.
Tuljapurkar. 2024. \href{https://doi.org/10.1111/ele.14551}{Density
dependence shapes life-history trade-offs in a food-limited population}.
Ecology letters 27:e14551.

\bibitem[\citeproctext]{ref-Kerr2009}
Kerr, B., and P. Godfrey-Smith. 2009.
\href{https://doi.org/10.1111/j.1558-5646.2008.00570.x}{{Generalization
of the price equation for evolutionary change}}. Evolution 63:531--536.

\bibitem[\citeproctext]{ref-Kitcher1993}
Kitcher, P. 1993. {The advancement of science}. Oxford University Press,
New York.

\bibitem[\citeproctext]{ref-Lande1982}
Lande, R. 1982. \href{https://doi.org/10.2307/1936778}{{A quantitative
genetic theory of life history evolution}}. Ecology 63:607--615.

\bibitem[\citeproctext]{ref-Lehtonen2016}
Lehtonen, J. 2016.
\href{https://doi.org/10.1016/j.tree.2016.07.006}{{Multilevel selection
in kin selection language}}. Trends in Ecology and Evolution xx:1--11.

\bibitem[\citeproctext]{ref-Lehtonen2018}
---------. 2018. \href{https://doi.org/10.1086/694891}{{The Price
equation, gradient dynamics, and continuous trait game theory}}.
American Naturalist 191:146--153.

\bibitem[\citeproctext]{ref-Lehtonen2020}
Lehtonen, J., S. Okasha, and H. Helanterä. 2020.
\href{https://doi.org/10.1098/rstb.2019.0350}{{Fifty years of the Price
equation}}. Philosophical Transactions of the Royal Society B:
Biological Sciences 375:20190350.

\bibitem[\citeproctext]{ref-Levins1966}
Levins, R. 1966. \href{https://doi.org/10.2307/27836590}{The strategy of
model building in population biology}. American Naturalist.

\bibitem[\citeproctext]{ref-Lion2018}
Lion, S. 2018. \href{https://doi.org/10.1086/694865}{{Theoretical
approaches in evolutionary ecology: environmental feedback as a unifying
perspective}}. American Naturalist 191.

\bibitem[\citeproctext]{ref-Lion2023}
Lion, S., A. Sasaki, and M. Boots. 2023.
\href{https://doi.org/10.1111/ele.14183}{Extending eco-evolutionary
theory with oligomorphic dynamics}. Ecology Letters 26:S22--S46.

\bibitem[\citeproctext]{ref-Loreau2001}
Loreau, M., and A. Hector. 2001.
\href{https://doi.org/10.1038/35083573}{{Partitioning selection and
complementarity in biodiversity experiments}}. Nature 412:72--76.

\bibitem[\citeproctext]{ref-Luque2016}
Luque, V. J. 2017. \href{https://doi.org/10.1007/s10539-016-9538-y}{{One
equation to rule them all: a philosophical analysis of the Price
equation}}. Biology and Philosophy 32:1--29.

\bibitem[\citeproctext]{ref-Luque2021}
Luque, V. J., and L. Baravalle. 2021.
\href{https://doi.org/10.1007/s11229-021-03339-6}{{The mirror of
physics: on how the price equation can unify evolutionary biology}}.
Synthese 199:12439--12462.

\bibitem[\citeproctext]{ref-MacCallum2012}
MacCallum, R. M., M. Mauch, A. Burt, and A. M. Leroi. 2012.
\href{https://doi.org/10.5061/dryad.h0228}{{Evolution of music by public
choice}}. Proceedings of the National Academy of Sciences
109:12081--12086.

\bibitem[\citeproctext]{ref-MacColl2011}
MacColl, A. D. C. 2011.
\href{https://doi.org/10.1016/j.tree.2011.06.009}{{The ecological causes
of evolution}}. Trends in Ecology and Evolution 26:514--522.

\bibitem[\citeproctext]{ref-Mateo-Tomas2017}
Mateo-Tomás, P., P. P. Olea, M. Moleón, N. Selva, and J. A.
Sánchez-Zapata. 2017. \href{https://doi.org/10.1111/geb.12673}{{Both
rare and common species support ecosystem services in scavenger
communities}}. Global Ecology and Biogeography 26:1459--1470.

\bibitem[\citeproctext]{ref-Mayr1960}
Mayr, E. 1959. \href{https://doi.org/10.1101/SQB.1959.024.01.003}{Where
are we? Genetics and twentieth century darwinism}. Pages 1--14
\emph{in}Cold spring harbor symposia on quantitative biology (Vol. 24).

\bibitem[\citeproctext]{ref-Morrison2000}
Morrison, M. 2000. Unifying scientific theories: Physical concepts and
mathematical structures. Cambridge University Press.

\bibitem[\citeproctext]{ref-Nagel1961}
Nagel, E. 1961. The structure of science: Problems in the logic of
scientific explanation. Harcourt, Brace \& World, New York, NY, USA.

\bibitem[\citeproctext]{ref-Odling-smee2003}
Odling-Smee, F. J., K. N. Laland, and M. W. Feldman. 2003. Niche
construction: The neglected process in evolution. Princeton University
Press.

\bibitem[\citeproctext]{ref-Okasha2006}
Okasha, S. 2006.
\href{https://doi.org/10.1093/acprof:oso/9780199267972.001.0001}{Evolution
and the levels of selection}. Oxford University Press.

\bibitem[\citeproctext]{ref-Okasha2020}
Okasha, S., and J. Otsuka. 2020.
\href{https://doi.org/10.1098/rstb.2019.0365}{{The Price equation and
the causal analysis of evolutionary change}}. Philosophical Transactions
of the Royal Society B: Biological Sciences 375:20190365.

\bibitem[\citeproctext]{ref-Page2002}
Page, K. M., and M. A. Nowak. 2002.
\href{https://doi.org/10.1006/jtbi.2002.3112}{Unifying evolutionary
dynamics}. Journal of theoretical biology 219:93--98.

\bibitem[\citeproctext]{ref-Patel2018}
Patel, S., M. H. Cortez, and S. J. Schreiber. 2018.
\href{https://doi.org/10.1101/104505}{{Partitioning the effects of
eco-evolutionary feedbacks on community stability}}. American Naturalist
191:1--29.

\bibitem[\citeproctext]{ref-Pelletier2009}
Pelletier, F., D. Garant, and A. P. Hendry. 2009.
\href{https://doi.org/10.1098/rstb.2009.0027}{Eco-evolutionary
dynamics}. Philosophical Transactions of the Royal Society London B
364:1483--1489.

\bibitem[\citeproctext]{ref-Pigliucci2009}
Pigliucci, M. 2009.
\href{https://doi.org/10.1111/j.1749-6632.2009.04578.x}{{An extended
synthesis for evolutionary biology}}. Annals of the New York Academy of
Sciences 1168:218--228.

\bibitem[\citeproctext]{ref-Price1970}
Price, G. R. 1970. \href{https://doi.org/10.1038/227520a0}{{Selection
and covariance}}.

\bibitem[\citeproctext]{ref-Price1972}
---------. 1972.
\href{https://doi.org/10.1111/j.1469-1809.1957.tb01874.x}{{Extension of
covariance selection mathematics}}. Annals of Human Genetics
35:485--490.

\bibitem[\citeproctext]{ref-Queller2017}
Queller, D. C. 2017. \href{https://doi.org/10.1086/690937}{{Fundamental
theorems of evolution}}. American Naturalist 189:000--000.

\bibitem[\citeproctext]{ref-Rees2016}
Rees, M., and S. P. Ellner. 2016.
\href{https://doi.org/10.1111/2041-210X.12487}{Evolving integral
projection models: Evolutionary demography meets eco-evolutionary
dynamics}. Methods in Ecology and Evolution 7:157--170.

\bibitem[\citeproctext]{ref-Rice2004}
Rice, S. H. 2004. {Evolutionary theory: mathematical and conceptual
foundations}. Sinauer Associates Incorporated.

\bibitem[\citeproctext]{ref-Rice2020}
---------. 2020.
\href{https://doi.org/10.1098/rstb.2019.0353}{{Universal rules for the
interaction of selection and transmission in evolution}}. Philosophical
Transactions of the Royal Society B: Biological Sciences 375.

\bibitem[\citeproctext]{ref-Rice2009}
Rice, S. H., and A. Papadopoulos. 2009.
\href{https://doi.org/10.1371/journal.pone.0007130}{{Evolution with
stochastic fitness and stochastic migration}}. PLoS One 4.

\bibitem[\citeproctext]{ref-Robertson1966}
Robertson, A. 1966. \href{https://doi.org/10.1017/S0003356100037752}{{A
mathematical model of the culling process in dairy cattle}}. Animal
Science 8:95--108.

\bibitem[\citeproctext]{ref-Roff2008}
Roff, D. A. 2008.
\href{https://doi.org/10.1007/s12041-008-0056-9}{{Defining fitness in
evolutionary models}}. Journal of Genetics 87:339--348.

\bibitem[\citeproctext]{ref-Simmonds2020}
Simmonds, E. G., E. F. Cole, B. C. Sheldon, and T. Coulson. 2020.
\href{https://doi.org/10.1111/ele.13603}{Phenological asynchrony: A
ticking time-bomb for seemingly stable populations?} Ecology Letters.
Blackwell Publishing Ltd.

\bibitem[\citeproctext]{ref-Turchin2001}
Turchin, P. 2001.
\href{https://doi.org/10.1034/j.1600-0706.2001.11310.x}{Does population
ecology have general laws?} Oikos 94:17--26.

\bibitem[\citeproctext]{ref-Ulrich2024}
Ulrich, W., N. J. Gotelli, G. Strona, and W. Godsoe. 2024.
\href{https://doi.org/10.1016/j.ecolmodel.2024.110695}{Reconsidering the
price equation: Benchmarking the analytical power of additive
partitioning in ecology}. Ecological Modelling 491:110695.

\bibitem[\citeproctext]{ref-VanVeelen2005}
van Veelen, M. 2005.
\href{https://doi.org/10.1016/j.jtbi.2005.04.026}{{On the use of the
Price equation}}. Journal of Theoretical Biology 237:412--426.

\bibitem[\citeproctext]{ref-VanVeelen2020}
---------. 2020. \href{https://doi.org/10.1098/rstb.2019.0355}{{The
problem with the Price equation}}. Philosophical Transactions of the
Royal Society B: Biological Sciences 375:20190355.

\bibitem[\citeproctext]{ref-VanVeelen2012}
van Veelen, M., J. García, M. W. Sabelis, and M. Egas. 2012.
\href{https://doi.org/10.1016/j.jtbi.2011.07.025}{{Group selection and
inclusive fitness are not equivalent; the Price equation vs. models and
statistics}}. Journal of Theoretical Biology 299:64--80.

\bibitem[\citeproctext]{ref-Winfree2015}
Winfree, R., J. W. Fox, N. M. Williams, J. R. Reilly, and D. P.
Cariveau. 2015. \href{https://doi.org/10.1111/ele.12424}{{Abundance of
common species, not species richness, drives delivery of a real-world
ecosystem service}}. Ecology Letters 18:626--635.

\bibitem[\citeproctext]{ref-Yamamichi2023}
Yamamichi, M., S. P. Ellner, and N. G. Hairston. 2023.
\href{https://doi.org/10.1111/ele.14197}{Beyond simple adaptation:
Incorporating other evolutionary processes and concepts into
eco-evolutionary dynamics}. Ecology Letters 26:S16--S21.

\end{CSLReferences}

\end{document}
