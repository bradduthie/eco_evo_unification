% Options for packages loaded elsewhere
\PassOptionsToPackage{unicode}{hyperref}
\PassOptionsToPackage{hyphens}{url}
\PassOptionsToPackage{dvipsnames,svgnames,x11names}{xcolor}
%
\documentclass[
]{article}
\usepackage{amsmath,amssymb}
\usepackage{lmodern}
\usepackage{setspace}
\usepackage{iftex}
\ifPDFTeX
  \usepackage[T1]{fontenc}
  \usepackage[utf8]{inputenc}
  \usepackage{textcomp} % provide euro and other symbols
\else % if luatex or xetex
  \usepackage{unicode-math}
  \defaultfontfeatures{Scale=MatchLowercase}
  \defaultfontfeatures[\rmfamily]{Ligatures=TeX,Scale=1}
\fi
% Use upquote if available, for straight quotes in verbatim environments
\IfFileExists{upquote.sty}{\usepackage{upquote}}{}
\IfFileExists{microtype.sty}{% use microtype if available
  \usepackage[]{microtype}
  \UseMicrotypeSet[protrusion]{basicmath} % disable protrusion for tt fonts
}{}
\makeatletter
\@ifundefined{KOMAClassName}{% if non-KOMA class
  \IfFileExists{parskip.sty}{%
    \usepackage{parskip}
  }{% else
    \setlength{\parindent}{0pt}
    \setlength{\parskip}{6pt plus 2pt minus 1pt}}
}{% if KOMA class
  \KOMAoptions{parskip=half}}
\makeatother
\usepackage{xcolor}
\IfFileExists{xurl.sty}{\usepackage{xurl}}{} % add URL line breaks if available
\IfFileExists{bookmark.sty}{\usepackage{bookmark}}{\usepackage{hyperref}}
\hypersetup{
  pdftitle={Darwin's Dream: Unifying ecological and evolutionary change},
  colorlinks=true,
  linkcolor={blue},
  filecolor={Maroon},
  citecolor={Blue},
  urlcolor={Blue},
  pdfcreator={LaTeX via pandoc}}
\urlstyle{same} % disable monospaced font for URLs
\usepackage[margin=1in]{geometry}
\usepackage{graphicx}
\makeatletter
\def\maxwidth{\ifdim\Gin@nat@width>\linewidth\linewidth\else\Gin@nat@width\fi}
\def\maxheight{\ifdim\Gin@nat@height>\textheight\textheight\else\Gin@nat@height\fi}
\makeatother
% Scale images if necessary, so that they will not overflow the page
% margins by default, and it is still possible to overwrite the defaults
% using explicit options in \includegraphics[width, height, ...]{}
\setkeys{Gin}{width=\maxwidth,height=\maxheight,keepaspectratio}
% Set default figure placement to htbp
\makeatletter
\def\fps@figure{htbp}
\makeatother
\setlength{\emergencystretch}{3em} % prevent overfull lines
\providecommand{\tightlist}{%
  \setlength{\itemsep}{0pt}\setlength{\parskip}{0pt}}
\setcounter{secnumdepth}{-\maxdimen} % remove section numbering
\newlength{\cslhangindent}
\setlength{\cslhangindent}{1.5em}
\newlength{\csllabelwidth}
\setlength{\csllabelwidth}{3em}
\newlength{\cslentryspacingunit} % times entry-spacing
\setlength{\cslentryspacingunit}{\parskip}
\newenvironment{CSLReferences}[2] % #1 hanging-ident, #2 entry spacing
 {% don't indent paragraphs
  \setlength{\parindent}{0pt}
  % turn on hanging indent if param 1 is 1
  \ifodd #1
  \let\oldpar\par
  \def\par{\hangindent=\cslhangindent\oldpar}
  \fi
  % set entry spacing
  \setlength{\parskip}{#2\cslentryspacingunit}
 }%
 {}
\usepackage{calc}
\newcommand{\CSLBlock}[1]{#1\hfill\break}
\newcommand{\CSLLeftMargin}[1]{\parbox[t]{\csllabelwidth}{#1}}
\newcommand{\CSLRightInline}[1]{\parbox[t]{\linewidth - \csllabelwidth}{#1}\break}
\newcommand{\CSLIndent}[1]{\hspace{\cslhangindent}#1}
\usepackage{fancyhdr}
\usepackage{lineno}
\linenumbers
\modulolinenumbers[2]
\ifLuaTeX
  \usepackage{selnolig}  % disable illegal ligatures
\fi

\title{Darwin's Dream: Unifying ecological and evolutionary change}
\author{A. Bradley Duthie\(^{1,a,*}\) and Victor J. Luque\(^{2,a}\)}
\date{{[}1{]} Department of Biological and Environmental Sciences,
University of Stirling, Stirling, Scotland {[}2{]} Department of
Philosophy, University of Valencia, Valencia, Spain {[}*{]}
Corresponding author:
\href{mailto:alexander.duthie@stir.ac.uk}{\nolinkurl{alexander.duthie@stir.ac.uk}}
{[}a{]} Equal contribution}

\begin{document}
\maketitle

\setstretch{2}
\hypertarget{abstract}{%
\section{Abstract}\label{abstract}}

\textbf{Biological evolution is realised through the same mechanisms of
birth and death that underlie change in population density.} \textbf{The
deep interdependence between ecology and evolution is well-established,
but much theory in each discipline has been developed in isolation.}
\textbf{While recent work has accomplished eco-evolutionary synthesis, a
gap remains between the logical foundations of ecology and evolution.}
\textbf{We bridge this gap with a new equation that defines a summed
value for a characteristic across individuals in a population, from
which the fundamental equations of population ecology and evolutionary
biology (the Price equation) are derived.} \textbf{We thereby unify the
fundamental equations of population ecology and biological evolution
under a general framework.} \textbf{Our unification further demonstrates
the equivalence between mean population growth rate and evolutionary
fitness, shows how ecological and evolutionary change are reflected in
the first and second statistical moments of fitness, respectively, and
links this change to ecosystem function.} \textbf{Finally, we outline
how our proposed framework can be used to unify social evolution and
density-dependent population growth.}

\textbf{Key words:} Ecology, Evolution, Eco-Evolutionary Theory,
Fundamental Theorem, Price Equation, Population Growth

\hypertarget{non-technical-summary}{%
\section{Non-technical Summary}\label{non-technical-summary}}

Biologists have long known that the ecology of population size is
inextricably linked with the evolution of population traits.
Nevertheless, the foundational theory underlying ecological and
evolutionary change have been historically separated, with recent
attempts to bridge this theoretical divide using eco-evolutionary
modelling. We discover a new equation from first principles that
provides a universal and complete definition for eco-evolutionary
change. This equation thereby offers a unified framework for ecology and
evolution, which also recovers the foundational theory of ecology and
evolution as developed in isolation.

\hypertarget{introduction}{%
\section{Introduction}\label{introduction}}

Theoretical unification of phenomena is a powerful tool for scientific
advancement. Such unification has been a major goal in scientific
research throughout history (\protect\hyperlink{ref-Kitcher1993}{Kitcher
1993}), and its value is perhaps most evident in reconciling unconnected
models and revealing new and unexpected empirical predictions. In
theoretical physics, Newton's unification of celestial and terrestrial
motion, and Maxwell's unification of electricity and magnetism, are
among the most important advances in scientific history. Likewise, in
the life sciences, the modern synthesis of natural selection and
Mendelian inheritance revolutionised how mechanisms underlying evolution
and genetics are understood and applied
(\protect\hyperlink{ref-Gayon2019}{Gayon and Huneman 2019}). The analogy
to physics is imperfect because evolution encompasses non-biological
systems with non-genetic inheritance (e.g.,
\protect\hyperlink{ref-Price1995}{Price 1995};
\protect\hyperlink{ref-Shennan2011}{Shennan 2011};
\protect\hyperlink{ref-MacCallum2012}{MacCallum et al. 2012};
\protect\hyperlink{ref-Lewens2015}{Lewens 2015}), but to illustrate our
point, we restrict our focus to biological evolution. Following the
modern synthesis, patterns in evolutionary biology and genetics could be
interpreted through a common framework.

Selection, mutation, recombination, migration, and drift all have
predictable consequences for both phenotypes and genotypes in a
population. These mechanisms are therefore foundational to understanding
evolution and genetics, and their roles in explaining evolutionary and
genetic phenomena are no more conceptually pluralistic than that of
gravity in explaining celestial and terrestrial motion. Instead,
biologists understand that each of these mechanisms will jointly affect
genetic composition and evolutionary change in a population. For
example, all else being equal, drift will lead to a loss of both
heterozygosity and adaptive phenotypes. This constrains the possible
relations that can exist between genetic composition and evolutionary
change and therefore clarifies the scope of hypotheses that are
biologically coherent (\protect\hyperlink{ref-Caswell1988}{Caswell
1988}). It would obviously be mistaken to model a population in which
drift had inconsistent effects on evolution and genetics, but such
mistakes would surely be common without the foundational theory required
to unify the two subjects. The absence of this theory is almost
unthinkable in modern biology. Evolution and genetics are arguably as
intertwined in biology as electricity and magnetism are in physics.

In contrast, theory underlying ecology and evolution has not been so
tightly intertwined. Until recently, population ecology and evolution in
particular have taken almost parallel paths for developing theory.
Although many revelatory models exist to connect ecological and
evolutionary phenomena (e.g., \protect\hyperlink{ref-Lion2018}{Lion
2018}; \protect\hyperlink{ref-Patel2018}{Patel, Cortez, and Schreiber
2018}; \protect\hyperlink{ref-Govaert2019}{Govaert et al. 2019}), the
foundational assumptions and equations of ecology and evolution in these
models remain mostly separated conceptually. But like the mechanisms
driving evolutionary and genetic change, those that drive ecological and
evolutionary change are well-established with predictable joint
consequences. In evolution, natural selection, drift, and gene flow are
all instantiated through a combination of different ecological factors
(\protect\hyperlink{ref-Connor2004}{Connor and Hartl 2004}). Most
notably, change in population density and phenotype are both caused by
individual birth and death (\protect\hyperlink{ref-Turchin2001}{Turchin
2001}). This should allow biologists to place ecological and
evolutionary change on a shared theoretical foundation.

We believe that such a foundation exists for population ecology and
evolutionary biology, and that it can bridge the fundamental equations
of both subjects and serve as a starting point for understanding change
in any population. Turchin (\protect\hyperlink{ref-Turchin2001}{2001})
argues for a simple birth and death model (Box 1) as a fundamental
equation of population ecology, and for exponential growth to
consequently be as fundamental to ecology as Newton's first law of
motion is to physics. Turchin
(\protect\hyperlink{ref-Turchin2001}{2001}) argues that general
principles are needed to establish a logical foundation for population
ecology. In evolutionary biology, the Price equation (Box 2) is almost
universally regarded as fundamental
(\protect\hyperlink{ref-Price1970}{Price 1970};
\protect\hyperlink{ref-Rice2004}{Rice 2004};
\protect\hyperlink{ref-Gardner2008}{Gardner 2008};
\protect\hyperlink{ref-Frank2015a}{Frank 2017};
\protect\hyperlink{ref-Luque2016}{Luque 2017};
\protect\hyperlink{ref-Luque2021}{Luque and Baravalle 2021}). It
exhaustively and exactly describes evolutionary change for any closed
evolving population (\protect\hyperlink{ref-Price1970}{Price 1970};
\protect\hyperlink{ref-Luque2016}{Luque 2017};
\protect\hyperlink{ref-Lehtonen2020}{Lehtonen, Okasha, and Helanterä
2020}), and it serves as the starting point for all other fundamental
equations in evolution (\protect\hyperlink{ref-Queller2017}{Queller
2017}). We believe that this logical foundation can be made sufficiently
broad to encompass both population and evolutionary change.

We propose an equation from which the fundamental ecological and
evolutionary equations can be derived. Derivation follows by adding
assumptions that are specific to population ecology or evolution in the
same way that key equations of population genetics (e.g., average
excess) or quantitative genetics (e.g., Breeder's equation) can be
derived from the Price equation
(\protect\hyperlink{ref-Queller2017}{Queller 2017}). We hope that our
equation can provide a starting point for building a shared logical
framework between ecology and evolution.

\begin{center}\rule{0.5\linewidth}{0.5pt}\end{center}

\begin{quote}
\textbf{Box 1:} The number of individuals in any closed population
(\(N\)) at any given time (\(t + 1\)) is determined by the existing
number of individuals (\(N_{t}\)), plus the number of births
(\(Births\)) minus the number of deaths (\(Deaths\)),
\[N_{t+1} = N_{t} + Births - Deaths.\] This equation is necessarily true
for any closed population. Despite its simplicity, it is a general
equation for defining population change and a starting point for
understanding population ecology. Turchin
(\protect\hyperlink{ref-Turchin2001}{2001}) notes that a consequence of
this fundamental equation is the tendency for populations to grow
exponentially (technically geometrically in the above case where time is
discrete). This inherent underlying tendency towards exponential growth
persists even as the complexities of real populations, such as
structure, stochasticity, or density-dependent effects are added to
population models (\protect\hyperlink{ref-Turchin2001}{Turchin 2001}).
Given the assumption that all individuals in the population are
identical, a per capita rate of birth, \(Births = bN_{t}\) and death,
\(Deaths = dN_{t}\) can be defined. Rearranging and defining
\(\lambda = 1 + b - d\) gives, \(N_{t+1} = N_{t}\lambda\). Here
\(\lambda\) is the finite rate of increase
(\protect\hyperlink{ref-Gotelli2001}{Gotelli 2001}), and note that
because \(0 \leq d \leq 1\), \(\lambda \geq 0\). Verbally, the change in
size of any closed population equals its existing size times its finite
rate of increase.
\end{quote}

\begin{center}\rule{0.5\linewidth}{0.5pt}\end{center}

\begin{center}\rule{0.5\linewidth}{0.5pt}\end{center}

\begin{quote}
\textbf{Box 2:} The Price equation is an abstract formula to represent
evolutionary change. Formulated originally in the early 1970s by George
Price (\protect\hyperlink{ref-Price1970}{Price 1970},
\protect\hyperlink{ref-Price1972}{1972}), it postulates some basic
properties that all evolutionary systems must satisfy: change over time,
ancestor and descendant relations, and a character or phenotype
(\protect\hyperlink{ref-Rice2004}{Rice 2004}). Using simple algebraic
language, the Price equation represents evolutionary change with the
predominant notation,
\[\bar{w}\Delta\bar{z} = \mathrm{Cov}\left(w, z\right) + \mathrm{E}\left(w\Delta z\right).\]
In the above equation, \(\Delta\bar{z}\) is the change in the average
character value \(z\) over a time step of arbitrary length, \(w\) is an
individual's absolute fitness, and \(\bar{w}\) average population
fitness. On the right-hand side of the equation, the first term is the
covariance between a character value \(z\) and fitness \(w\), which
reflects \(\bar{z}\) change attributable to differential survival and
reproduction. The second term is the expected value of \(\Delta z\),
which reflects the extent to which offspring deviate from parents in
\(z\) (\protect\hyperlink{ref-Rice2004}{Rice 2004};
\protect\hyperlink{ref-Okasha2006}{Okasha 2006};
\protect\hyperlink{ref-Frank2012}{Frank 2012}). A more specific version
of the covariance term was already known within the quantitative and
population genetics tradition
(\protect\hyperlink{ref-Robertson1966}{Robertson 1966}), usually
representing the action of natural selection. The Price equation adds an
expectation term and abstracts away from any specific mechanisms of
replication or reproduction, or mechanisms of inheritance. Its
definitional nature and lack of substantive biological assumptions has
been portrayed both as a strength (\protect\hyperlink{ref-Rice2004}{Rice
2004}; \protect\hyperlink{ref-Frank2012}{Frank 2012};
\protect\hyperlink{ref-Luque2016}{Luque 2017};
\protect\hyperlink{ref-Baravalle2022}{Baravalle and Luque 2022}), and
its greatest weakness. The abstract nature of the Price equation places
it at the top of the hierarchy of fundamental theorems of evolution from
which the rest (Robertson's theorem, Fischer's fundamental theorem,
breeder's equation, Hamilton's rule, adaptive dynamics, etc.) can be
easily derived (\protect\hyperlink{ref-Queller2017}{Queller 2017};
\protect\hyperlink{ref-Lehtonen2016}{Lehtonen 2016},
\protect\hyperlink{ref-Lehtonen2018}{2018}). This abstractness is also
key to developing a more general view of evolution
(\protect\hyperlink{ref-Rice2020}{Rice 2020};
\protect\hyperlink{ref-Luque2021}{Luque and Baravalle 2021};
\protect\hyperlink{ref-Edelaar2023}{Edelaar, Otsuka, and Luque 2023}).
In contrast, some researchers consider the Price equation just a
triviality (even tautological), and useless without further modelling
assumptions (\protect\hyperlink{ref-VanVeelen2005}{van Veelen 2005};
\protect\hyperlink{ref-VanVeelen2012}{van Veelen et al. 2012}). The
debate remains open (\protect\hyperlink{ref-VanVeelen2020}{van Veelen
2020}; \protect\hyperlink{ref-Baravalle2024}{Baravalle et al. 2024}).
\end{quote}

\begin{center}\rule{0.5\linewidth}{0.5pt}\end{center}

\hypertarget{a-foundation-for-biological-evolution-and-population-ecology}{%
\section{A foundation for biological evolution and population
ecology}\label{a-foundation-for-biological-evolution-and-population-ecology}}

To fully unify biological evolution and population ecology, we must
reconcile the general equation for population change (Box 1) with the
Price equation (Box 2). Previous theory has produced mathematical
synthesis between ecological and evolutionary equations (e.g.,
\protect\hyperlink{ref-Page2002}{Page and Nowak 2002};
\protect\hyperlink{ref-Collins2009}{Collins and Gardner 2009}), but not
established a unified conceptual framework for joint ecological and
evolutionary change. For example, Page and Nowak
(\protect\hyperlink{ref-Page2002}{2002}) demonstrate a mathematical
equivalence between the Price equation and a general equation for
community dynamics. In both equations, the relative frequencies of
different ecological (species) or evolutionary (phenotype) variants map
to their respective ecological (population growth) or evolutionary
(fitness) change in a way that is mathematically equivalent. Collins and
Gardner (\protect\hyperlink{ref-Collins2009}{2009}) use the Price
equation to build a general quantitative framework that partitions
phenotypic change into ecological, evolutionary, and physiological
components. The average change in a community-wide trait (e.g., carbon
uptake) is exactly and generally partitioned into change attributable to
changes in species composition, expected evolution of lineages within
species, and expected change in lineage physiology across all species.
Both Page and Nowak (\protect\hyperlink{ref-Page2002}{2002}) and Collins
and Gardner (\protect\hyperlink{ref-Collins2009}{2009}) use relative
species abundances, meaning that species frequencies must sum to one.
This is useful for demonstrating mathematical equivalence
(\protect\hyperlink{ref-Page2002}{Page and Nowak 2002}) or expected
change in a community-wide trait
(\protect\hyperlink{ref-Collins2009}{Collins and Gardner 2009}), but a
different approach is needed to recover absolute change in population
size and thereby recover a joint foundation for ecology and evolution.

The Price equation is critical for partitioning different components of
biological change (\protect\hyperlink{ref-Price1970}{Price 1970};
\protect\hyperlink{ref-Frank1997}{Frank 1997};
\protect\hyperlink{ref-Gardner2008}{Gardner 2008};
\protect\hyperlink{ref-Luque2016}{Luque 2017};
\protect\hyperlink{ref-Lehtonen2018}{Lehtonen 2018};
\protect\hyperlink{ref-Queller2017}{Queller 2017}). It has also been
highly useful for integrating evolutionary theory across disciplines
(\protect\hyperlink{ref-Fox2006}{Fox 2006};
\protect\hyperlink{ref-Brantingham2007}{Brantingham 2007};
\protect\hyperlink{ref-MacCallum2012}{MacCallum et al. 2012};
\protect\hyperlink{ref-Frank2015}{Frank 2015};
\protect\hyperlink{ref-Godsoe2021}{Godsoe et al. 2021};
\protect\hyperlink{ref-Ulrich2024}{Ulrich et al. 2024}). These
properties would seem to make it an intuitive starting point for a
logical foundation of ecology and evolution, perhaps through some kind
of mathematical equivalence (e.g., \protect\hyperlink{ref-Page2002}{Page
and Nowak 2002}) or addition of terms (e.g.,
\protect\hyperlink{ref-Collins2009}{Collins and Gardner 2009}), or
through the use of its recursive structure (e.g.,
\protect\hyperlink{ref-Kerr2009}{Kerr and Godfrey-Smith 2009};
\protect\hyperlink{ref-Frank2012}{Frank 2012}). But despite its
flexibility, the Price equation still relies on relative frequencies,
which must by definition sum to one
(\protect\hyperlink{ref-Frank2015}{Frank 2015}). This is because the
Price equation describes the average change in a population; the
frequency of entities is scaled thereby conserving total probability
(\protect\hyperlink{ref-Frank2015}{Frank 2015},
\protect\hyperlink{ref-Frank2016}{2016}). But to recover the fundamental
principle of exponential population growth
(\protect\hyperlink{ref-Turchin2001}{Turchin 2001}), this scaling must
be avoided in a fundamental equation of ecology and evolution.

We therefore begin with the most fundamental axioms underlying the
ecology and evolution of living systems
(\protect\hyperlink{ref-Rice2004}{Rice 2004};
\protect\hyperlink{ref-Rice2009}{Rice and Papadopoulos 2009}). In such
systems, diversity is discontinuous and can be defined in terms of
discrete entities (\protect\hyperlink{ref-Dobzhansky1970}{Dobzhansky
1970}). Our framework is general enough that entities can be anything
discrete, but we will focus on each entity \(i\) as an individual
organism. Individuals give rise to new individuals through birth such
that \(\beta_{i}\) is the number of births attributable to \(i\).
Individuals are removed from the population through death such that
\(\delta_{i}\) is an indicator variable that takes a value of 1 (death
of \(i\)) or 0 (persistence of \(i\)). All individuals are defined by
some characteristic \(z_{i}\), and \(\Delta z_{i}\) defines any change
in \(z_{i}\) from one time step \(t\) to the next \(t + 1\). The total
number of individuals in the population is \(N\). From this foundation,
we can define \(\Omega\) to be a summed characteristic across \(N\)
entities,

\[\Omega = \sum_{i=1}^{N} \left(\beta_{i} - \delta_{i} + 1 \right)\left(z_{i} + \Delta z_{i} \right).
\tag{1}
\]

From eqn 1, we can derive the most fundamental equations of population
ecology (Box 1) and evolutionary biology (Box 2) through an appropriate
interpretation of \(z_{i}\).

\hypertarget{population-ecology}{%
\section{Population ecology}\label{population-ecology}}

To recover the general equation for population ecology (Box 1), we
define \(z_{i}\) as the identity of \(i\) belonging to the population.
In other words, we set \(z_{i} = 1\) to simply indicate that \(i\) is a
member of the population. Further, we assume that individuals do not
change species by setting \(\Delta z_{i} = 0\). In this case,

\[\Omega = \sum_{i=1}^{N} \left(\beta_{i} - \delta_{i} + 1 \right).\]

Since we assume that individuals are identical, we can drop the
subscript \(i\) such that \(\beta_{i} = \beta\) and
\(\delta_{i} = \delta\). Summing from 1 to \(N\), we can rewrite the
above,

\[\Omega = N\left(1 + \beta - \delta \right).\]

We can now interpret \(\Omega\) as the population size at \(t+1\),
\(N_{t+1}\) and note that \(N\beta\), \(N\delta\), and \(N\) are the
total births, total deaths, and size of the population at \(t\),
respectively,

\[N_{t+1} = N_{t} + Births - Deaths.
\tag{2}
\]

If we define \(\lambda = 1 + \beta - \delta\) (Box 1), then we can
rewrite,

\[N_{t+1} = N_{t}\lambda.
\tag{3}
\]

We therefore recover the general equation for population ecology (eqn 2)
and the fundamental property of exponential growth in populations (eqn
3, \protect\hyperlink{ref-Turchin2001}{Turchin 2001}).

\hypertarget{evolutionary-biology}{%
\section{Evolutionary biology}\label{evolutionary-biology}}

Recovering the Price equation requires a few more steps. We start by
defining individual fitness,

\[w_{i} = \beta_{i} - \delta_{i} + 1.
\tag{4}
\]

In this definition, the longevity of the individual matters. An
individual that survives from \(t\) to \(t + 1\) has a higher fitness
than one that dies, even if both have the same reproductive output. With
this definition of fitness (eqn 4), we substitute,

\[\Omega = \sum_{i=1}^{N} \left(w_{i}z_{i} + w_{i}\Delta z_{i} \right).
\tag{5}
\]

We can break eqn 5 down further and multiply each side by \(1/N\),

\[\frac{1}{N}\Omega = \frac{1}{N}\sum_{i=1}^{N} \left(w_{i}z_{i} \right) + \frac{1}{N}\sum_{i=1}^{N}\left( w_{i}\Delta z_{i} \right).
\tag{6}
\]

We can rewrite the terms on the right-hand side of eqn 6 as expected
values and remove the subscripts,

\[\frac{1}{N}\Omega = \mathrm{E}\left(w z \right) + \mathrm{E}\left( w \Delta z  \right).
\tag{7}
\]

Now we must consider the total conservation of probability
(\protect\hyperlink{ref-Frank2015}{Frank 2015},
\protect\hyperlink{ref-Frank2016}{2016}). In eqn 7, \(\Omega\) is the
total sum trait values (\(z_{i}\)) across the entire population at
\(t + 1\) divided by the number of individuals (\(N\)) in the population
at \(t\). But the size of the population can change from \(t\) to
\(t + 1\). To recover mean trait change for the Price equation (and
therefore conserve total probability), we need to account for this
change in population size. The left-hand side of eqn 7 describes
contributions to the sum trait value from the new population at
\(t + 1\). But we cannot treat \(\Omega/N\) as the mean of \(z\) at
\(t+1\) (\(\bar{z}'\)) because we need to weigh \(N\) by the mean
fitness of the population at \(t\) to account for any change in
population size from \(t\) to \(t+1\).

For example, if mean fitness at \(t\) is 2 (i.e., \(\bar{w} = 2\)), then
half as many individuals will have contributed to \(\Omega\) in
\(t + 1\) than would have if \(\bar{w} = 1\) (i.e., there are \(N\)
individuals at \(t\) and \(2N\) individuals at \(t + 1\)). We therefore
need to multiply the mean trait value \(\bar{z}'\) (at \(t + 1\)) by the
mean fitness \(\bar{w}\) (at \(t\)) to recover the mean contribution of
the \(N\) individuals at \(t\) to the total \(\Omega\). Consequently,

\[\Omega = N\bar{w}\bar{z}'
\tag{8}
\]

Equation 8 conserves the total probability and now clarifies \(\Omega\)
as a summed trait value, which equals expected population growth at
\(t\) times mean trait value at \(t + 1\). This is consistent with the
population ecology derivation from the previous section where
\(z_{i} = 1\) by definition, and \(\Omega = N_{t+1}\) (note
\(\lambda = \bar{w}\)). We can therefore rewrite eqn 7,

\[\bar{w}\bar{z}' = \mathrm{E}\left(w z \right) + \mathrm{E}\left( w \Delta z  \right).
\tag{9}
\]

We can rearrange eqn 9 to derive the Price equation by expressing
covariance as,
\(\mathrm{Cov}(X,Y) = \mathrm{E}(XY) - \mathrm{E}(X)\mathrm{E}(Y)\), and
therefore
\(\mathrm{E}(XY) = \mathrm{Cov}(X,Y) + \mathrm{E}(X)\mathrm{E}(Y)\).
Substituting into eqn 9,

\[\bar{w}\bar{z}' = \mathrm{Cov}\left(w ,z \right) + \bar{w}\bar{z} + \mathrm{E}\left( w \Delta z  \right).\]

From here,

\[\bar{w}\left(\bar{z}' - \bar{z}\right) = \mathrm{Cov}\left(w ,z \right) + \mathrm{E}\left( w \Delta z  \right).\]

Since \(\Delta \bar{z} = \left(\bar{z}' - \bar{z}\right)\),

\[\bar{w}\Delta \bar{z} = \mathrm{Cov}\left(w ,z \right) + \mathrm{E}\left( w \Delta z  \right).
\tag{10}
\]

From eqn 1, which describes fundamental birth and death processes in a
population, we can derive both the most fundamental model of population
ecology (eqn 2; Box 1) and the fundamental equation of evolution (eqn
10; Box 2). We might also consider \(\Omega\) to be a quantity of
ecosystem function, in that it is the summed effect of the traits
(\(z_{i}\)) of individual entities (\(i\)) in the population. For
example, \(z\) might be interpreted as biomass or photosynthetic rate.

\hypertarget{discussion}{%
\section{Discussion}\label{discussion}}

A classical sign of scientific progress is the ability to connect
disparate theories and models to show how empirical and theoretical
models are logical (mathematical) consequences of more fundamental ones
(\protect\hyperlink{ref-Nagel1961}{Nagel 1961}). Conceptual unification
therefore has a critical role in advancing scientific theory
(\protect\hyperlink{ref-Morrison2000}{Morrison 2000}). Showing how
apparently disparate phenomena follow from the same shared principles
reveals what is most fundamental about the natural world and provides a
foundation for further scientific investigation and the construction of
more coherent and predictive models. Our eqn 1 provides a foundation for
the conceptual unification of the fundamental equations of population
ecology and evolution.

Rather than making simplifying assumptions, as is the approach for most
ecological and evolutionary models, we focus on fundamental axioms that
are universal to closed biological systems: discrete individuals, birth,
death, and change over time. We define an abstract sum (\(\Omega\)), to
which all individuals within the population contribute. From the simple
assumptions of population identity (\(z_{i} = 1\)) and invariability
(\(\Delta z_{i} = 0\)), we recover the most general equation of
population ecology (Box 1) and principle of exponential growth
(\(N_{t+1} = N_{t}\lambda\)). By defining individual fitness (\(w_{i}\))
and applying the total conservation of probability to individual
frequencies (\protect\hyperlink{ref-Frank2015}{Frank 2015},
\protect\hyperlink{ref-Frank2016}{2016}), we recover the most
fundamental equation of evolution (Box 2). We therefore deliver a
general framework to unify fundamental equations of ecological and
evolutionary change.

The Price equation provides a complete and exact description of
evolution in any closed evolving system
(\protect\hyperlink{ref-Price1970}{Price 1970};
\protect\hyperlink{ref-Frank2012}{Frank 2012}). It is derived by
rearranging the mathematical notation defining changes in the
frequencies and characteristics of any type of entity (e.g.,
individuals, alleles, \protect\hyperlink{ref-Price1970}{Price 1970};
\protect\hyperlink{ref-Gardner2008}{Gardner 2008};
\protect\hyperlink{ref-Luque2016}{Luque 2017}). This derivation
partitions total characteristic change into different components, making
it possible to isolate evolutionary mechanisms (e.g., selection) and
levels of biological organisation (e.g., group, individual,
\protect\hyperlink{ref-Frank1995}{Frank 1995},
\protect\hyperlink{ref-Frank2012}{2012};
\protect\hyperlink{ref-Kerr2009}{Kerr and Godfrey-Smith 2009};
\protect\hyperlink{ref-Luque2016}{Luque 2017};
\protect\hyperlink{ref-Okasha2020}{Okasha and Otsuka 2020}). Because of
its abstract nature and lack of any system-specific assumptions, the
Price equation makes no predictions about what will happen in any
particular system (\protect\hyperlink{ref-Gardner2020}{Gardner 2020}).
Its role is not to predict, but to formally and completely define and
separate components of evolutionary change. The same is true of the
general equation for population change (eqns 2 and 3), at least as we
have used it here where it serves to define what population growth means
in ecology. This equation formally and completely describes population
change in terms of births and deaths. In eqn 1, we therefore have a
fundamental equation from which we can derive complete ecological and
evolutionary change in any closed biological population. We anticipate
that this will be useful for eco-evolutionary theory in the same way
that the Price equation is useful for evolutionary theory: facilitating
specific model development and identifying new conceptual insights,
unresolved errors, and sources of model disagreements.

Our unification demonstrates the equivalence between the finite rate of
increase \(\lambda\) (Box 1, \protect\hyperlink{ref-Gotelli2001}{Gotelli
2001}) and population mean evolutionary fitness \(\bar{w}\) (Box 2). The
population growth equation \(N_{t+1} = N_{t}\lambda\) could therefore be
rewritten as \(N_{t+1} = N_{t}\bar{w}\). We do not claim to be the first
to notice this equivalence between population growth and mean population
fitness. Indeed, the relationship between population growth rate and
evolutionary fitness has been proposed and applied many times before
(e.g., \protect\hyperlink{ref-Fisher1930}{Fisher 1930};
\protect\hyperlink{ref-Charlesworth1980}{Charlesworth 1980};
\protect\hyperlink{ref-Lande1982}{Lande 1982};
\protect\hyperlink{ref-Roff2008}{Roff 2008};
\protect\hyperlink{ref-Lion2018}{Lion 2018}). For example, Lande
(\protect\hyperlink{ref-Lande1982}{1982}) explicitly concludes mean
absolute fitness per unit time is \(\bar{w} = e^{r}\), where
\(r = ln(\lambda)\). We show this from first principles and clarify the
relationship between fitness and population growth. Over an arbitrary
length of time, fitness is properly defined as
\(w_{i} = \beta_{i} - \delta_{i} + 1\). Over an individual's lifetime
(which, by definition, includes death), fitness is therefore
\(\beta_{i}\). Interestingly, the rate of change in ecology and
evolution are reflected in the first and second statistical moments of
fitness, respectively. Population growth rate reflects mean fitness
\(\bar{w}\), while the rate of evolutionary change reflects the variance
in fitness \(\mathrm{Var}(w)\) (i.e., Fisher's fundamental theorem,
\protect\hyperlink{ref-Frank1997}{Frank 1997};
\protect\hyperlink{ref-Rice2004}{Rice 2004};
\protect\hyperlink{ref-Queller2017}{Queller 2017}).

Our unification may also help explain, at least partially, some of the
success of classical population genetic models. For decades, population
genetics (and to some extent quantitative genetics) has been accused of
being a reductionistic view of evolution, reducing everything to changes
in allele frequencies and abstracting away from individuals and their
environments (e.g., the ecological interactions,
\protect\hyperlink{ref-MacColl2011}{MacColl 2011}). This has been a line
of argumentation by some defenders of the so-called Extended Synthesis
(\protect\hyperlink{ref-Pigliucci2009}{Pigliucci 2009}), especially in
relation to niche construction
(\protect\hyperlink{ref-Odling-smee2003}{Odling-Smee, Laland, and
Feldman 2003}). Famously, Mayr (\protect\hyperlink{ref-Mayr1960}{1959})
characterized population genetics as a simple input and output of genes,
analogous to ``the adding of certain beans to a beanbag and the
withdrawing of others'' (also called ``beanbag genetics'').

Historical critics of population genetics could not articulate a clear
explanation for why it works so well despite all of its idealisations
and simplifications. From the Price equation, we are able to recover
classical population and quantitative genetic models
(\protect\hyperlink{ref-Queller2017}{Queller 2017}) and develop new ones
(\protect\hyperlink{ref-Rice2004}{Rice 2004},
\protect\hyperlink{ref-Rice2020}{2020};
\protect\hyperlink{ref-Luque2016}{Luque 2017};
\protect\hyperlink{ref-Lion2018}{Lion 2018}). Our eqn 1 contains ecology
at its core, and we show how the Price equation logically follows from
it after accounting for absolute population growth (eqn 8). We therefore
conclude that population and quantitative genetic equations contain
ecology (no matter how hidden), and the ecological nature of evolution
is implicit in population and quantitative genetic models.

We have focused on the dynamics of a closed population, and in doing so
leave ecological and evolutionary change attributable to migration for
future work. In population ecology, immigration and emigration can be
incorporated by adding a term for each to the right-hand side of the
equation in Box 1 (\protect\hyperlink{ref-Gotelli2001}{Gotelli 2001}).
In evolution, because the Price equation relies on mapping
ancestor-descendant relationships, accounting for migration is more
challenging. Kerr and Godfrey-Smith
(\protect\hyperlink{ref-Kerr2009}{2009}) demonstrate how the Price
equation can be extended to allow for arbitrary links between ancestors
and descendants, thereby extending the Price equation to allow for
immigration and emigration. Frank
(\protect\hyperlink{ref-Frank2012}{2012}) presents a simplified version
of Kerr and Godfrey-Smith (\protect\hyperlink{ref-Kerr2009}{2009}) that
allows some fraction of descendants to be unconnected to ancestors. In
both ecology and evolution, accounting for migration is done through the
use of additional terms on the right-hand side of the equations.

Our fundamental equation is complete and exact. It therefore implicitly
includes any effects of density dependence on population growth (see Box
1), or any social effects on evolutionary change (see Box 2). We can
make both of these effects explicit with the same partition in the
summation on the right-hand side of eqn 1. To do this, we multiply the
expression
\(\left(\beta_{i} - \delta_{i} + 1 \right)\left(z_{i} + \Delta z_{i} \right)\)
by
\(\left(1 - \sum_{j=1}^{N} a_{ij}\left(z_{j} + \Delta z_{j}\right)\right)\),
where \(j\) indexes the same individuals as \(i\), and \(a_{ij}\)
defines the effect \(j\) has on the contribution of \(i\) to \(\Omega\).
In evolutionary terms, \(a_{ij}\) defines how \(j\) modulates the
fitness of \(i\) (\(w_{i}\)). In ecological terms, \(a_{ij}\) defines
how \(j\) modulates the contribution of \(i\) to total population
growth. If we define \(\alpha\) as the summed effect of all individuals
on a focal \(i\) (i.e., \(\alpha = \sum_{j=1}^{N} a_{ij}\)), then we can
partition ecological change into density-independent and
density-dependent effects and derive a discrete time logistic growth
equation. We save these derivations for a future investigations focused
on evolutionary and community ecology.

We have shown that we can derive the fundamental equations of population
ecology and biological evolution from a single unifying equation.
Lastly, we propose our eqn 1 as a potential starting point for defining
ecosystem function and further conceptual unification between ecology,
evolution, and ecosystem function. The Price equation has previously
been used to investigate ecosystem function
(\protect\hyperlink{ref-Loreau2001}{Loreau and Hector 2001};
\protect\hyperlink{ref-Fox2006}{Fox 2006}), but not with any attempt
towards conceptual unification with evolutionary biology. For example,
Fox (\protect\hyperlink{ref-Fox2006}{2006}) applied the abstract
properties of the Price equation to partition total change in ecosystem
function into separate components attributable to species richness,
species composition, and context dependent effects. This approach
provides a framework for comparing the effects of biodiversity on
ecosystem function in empirical systems (e.g.,
\protect\hyperlink{ref-Fox2006}{Fox 2006};
\protect\hyperlink{ref-Winfree2015}{Winfree et al. 2015};
\protect\hyperlink{ref-Mateo-Tomas2017}{Mateo-Tomás et al. 2017}).
Instead, our eqn 1 defines \(\Omega\) as total ecosystem function
contributed by a focal population (e.g., biomass production,
decomposition, photosynthetic rate). It is therefore possible to
investigate ecological, evolutionary, and ecosystem function change from
the same shared framework.

\hypertarget{acknowledgements}{%
\section{Acknowledgements}\label{acknowledgements}}

This manuscript was supported by joint funding between the French
Foundation for Research on Biodiversity (FRB) Centre for the Synthesis
and Analysis of Biodiversity (CESAB) and the German Centre for
Integrative Biodiversity Research (sDiv). It was written as part of the
Unification of Modern Coexistence Theory and Price Equation (UNICOP)
project. Victor J. Luque was also supported by the Spanish Ministry of
Science and Innovation (Project: PID2021-128835NB-I00), and the
Conselleria d'Innovaci\(\'{o}\), Universitats, Ci\(\`{e}\)ncia i
Societat Digital -- Generalitat Valenciana (Project: CIGE/2023/16). We
are grateful for many conversations with S\(\'{e}\)bastien Lion, Kelsey
Lyberger, Swati Patel, and especially Lynn Govaert, whose questions
helped us clarify the relationship between population growth and
fitness. Brad Duthie would also like to thank Brent Danielson and Stan
Harpole. Victor J. Luque would also like to thank Lorenzo Baravalle, Pau
Carazo, Santiago Ginnobili, Manuel Serra, and Ariel Roff\(\'{e}\).

\hypertarget{author-contributions}{%
\section{Author Contributions}\label{author-contributions}}

Both authors came up with the question idea. Following many discussions
between the authors, ABD proposed the initial equation with subsequent
exploration and development from both authors. Both authors contributed
to the writing.

\hypertarget{competing-interests}{%
\section{Competing Interests}\label{competing-interests}}

The authors declare no competing interests.

\hypertarget{references}{%
\section*{References}\label{references}}
\addcontentsline{toc}{section}{References}

\hypertarget{refs}{}
\begin{CSLReferences}{1}{0}
\leavevmode\vadjust pre{\hypertarget{ref-Baravalle2022}{}}%
Baravalle, Lorenzo, and Victor J. Luque. 2022. {``Towards a Pricean
Foundation for Cultural Evolutionary Theory.''} \emph{Theoria} 37 (2):
209--31. \url{https://doi.org/10.1387/theoria.21940}.

\leavevmode\vadjust pre{\hypertarget{ref-Baravalle2024}{}}%
Baravalle, Lorenzo, Ariel Roffé, Victor J. Luque, and Santiago
Ginnobili. 2024. {``The Value of Price.''} In \emph{Biological Theory}.
\url{https://doi.org/10.1007/s13752-024-00482-4}.

\leavevmode\vadjust pre{\hypertarget{ref-Brantingham2007}{}}%
Brantingham, P. Jeffrey. 2007. {``{A unified evolutionary model of
archaeological style and function based on the Price equation}.''}
\emph{American Antiquity} 72 (3): 395--416.
\url{https://doi.org/10.2307/40035853}.

\leavevmode\vadjust pre{\hypertarget{ref-Caswell1988}{}}%
Caswell, Hal. 1988. {``Theory and Models in Ecology: A Different
Perspective.''} \emph{Ecological Modelling} 43 (1-2): 33--44.
\url{https://doi.org/10.1016/0304-3800(88)90071-3}.

\leavevmode\vadjust pre{\hypertarget{ref-Charlesworth1980}{}}%
Charlesworth, Brian. 1980. \emph{Evolution in Age-Structured
Populations}. Cambridge Studies in Mathematical Biology. Cambridge:
Cambridge University Press.

\leavevmode\vadjust pre{\hypertarget{ref-Collins2009}{}}%
Collins, Sinéad, and Andy Gardner. 2009. {``{Integrating physiological,
ecological and evolutionary change: A Price equation approach}.''}
\emph{Ecology Letters} 12 (8): 744--57.
\url{https://doi.org/10.1111/j.1461-0248.2009.01340.x}.

\leavevmode\vadjust pre{\hypertarget{ref-Connor2004}{}}%
Connor, J., and Daniel L. Hartl. 2004. \emph{{A premier of ecological
genetics}}. Sinauer Associates Incorporated.

\leavevmode\vadjust pre{\hypertarget{ref-Dobzhansky1970}{}}%
Dobzhansky, Theodosius. 1970. \emph{Genetics of the Evolutionary
Process}. Vol. 139. Columbia University Press.

\leavevmode\vadjust pre{\hypertarget{ref-Edelaar2023}{}}%
Edelaar, Pim, Jun Otsuka, and Victor J. Luque. 2023. {``{A generalised
approach to the study and understanding of adaptive evolution}.''}
\emph{Biological Reviews} 98 (1): 352--75.
\url{https://doi.org/10.1111/brv.12910}.

\leavevmode\vadjust pre{\hypertarget{ref-Fisher1930}{}}%
Fisher, Ronald A. 1930. \emph{The Genetical Theory of Natural
Selection}. Oxford University Press, Oxford, UK.

\leavevmode\vadjust pre{\hypertarget{ref-Fox2006}{}}%
Fox, Jeremy W. 2006. {``{Using the price equation to partition the
effects of biodiversity loss on ecosystem function}.''} \emph{Ecology}
87 (11): 2687--96.
\url{https://doi.org/10.1890/0012-9658(2006)87\%5B2687:utpetp\%5D2.0.co;2}.

\leavevmode\vadjust pre{\hypertarget{ref-Frank1995}{}}%
Frank, Steven A. 1995. {``{George Price's contributions to evolutionary
genetics}.''} \emph{Journal of Theoretical Biology} 175: 373--88.
\url{https://doi.org/10.1006/jtbi.1995.0148}.

\leavevmode\vadjust pre{\hypertarget{ref-Frank1997}{}}%
---------. 1997. {``{The Price equation, Fisher's fundamental theorem,
kin selection, and causal analysis}.''} \emph{Evolution} 51 (6):
1712--29. \url{https://doi.org/10.1111/j.1558-5646.1997.tb05096.x}.

\leavevmode\vadjust pre{\hypertarget{ref-Frank2012}{}}%
---------. 2012. {``{Natural selection. IV. The Price equation}.''}
\emph{Journal of Evolutionary Biology} 25: 1002--19.
\url{https://doi.org/10.1111/j.1420-9101.2012.02498.x}.

\leavevmode\vadjust pre{\hypertarget{ref-Frank2015}{}}%
---------. 2015. {``{D'Alembert's direct and inertial forces acting on
populations: The price equation and the fundamental theorem of natural
selection}.''} \emph{Entropy} 17 (10): 7087--7100.
\url{https://doi.org/10.3390/e17107087}.

\leavevmode\vadjust pre{\hypertarget{ref-Frank2016}{}}%
---------. 2016. {``{Common probability patterns arise from simple
invariances}.''} \emph{Entropy} 18 (5): 1--22.
\url{https://doi.org/10.3390/e18050192}.

\leavevmode\vadjust pre{\hypertarget{ref-Frank2015a}{}}%
---------. 2017. {``{Universal expressions of population change by the
Price equation: natural selection, information, and maximum entropy
production}.''} \emph{Ecology and Evolution}, no. February: 1--16.
\url{https://doi.org/10.1002/ece3.2922}.

\leavevmode\vadjust pre{\hypertarget{ref-Gardner2008}{}}%
Gardner, Andy. 2008. {``{The Price equation}.''} \emph{Current Biology}
18 (5): 198--202. \url{https://doi.org/10.1016/j.cub.2008.01.005}.

\leavevmode\vadjust pre{\hypertarget{ref-Gardner2020}{}}%
---------. 2020. {``{Price's equation made clear}.''}
\emph{Philosophical Transactions of the Royal Society B: Biological
Sciences} 375 (1797): 20190361.
\url{https://doi.org/10.1098/rstb.2019.0361}.

\leavevmode\vadjust pre{\hypertarget{ref-Gayon2019}{}}%
Gayon, Jean, and Philippe Huneman. 2019. {``The Modern Synthesis:
Theoretical or Institutional Event?''} \emph{Journal of the History of
Biology} 52 (4): 519--35.
\url{https://doi.org/10.1007/s10739-019-09569-2}.

\leavevmode\vadjust pre{\hypertarget{ref-Godsoe2021}{}}%
Godsoe, William, Katherine E. Eisen, Daniel Stanton, and Katherine M.
Sirianni. 2021. {``Selection and Biodiversity Change.''}
\emph{Theoretical Ecology} 14 (3): 367--79.
\url{https://doi.org/10.1007/s12080-020-00478-3}.

\leavevmode\vadjust pre{\hypertarget{ref-Gotelli2001}{}}%
Gotelli, Nicholas J. 2001. {``A Primer of Ecology. Sinauer Associate.''}
\emph{Inc. Sunderland, MA}.

\leavevmode\vadjust pre{\hypertarget{ref-Govaert2019}{}}%
Govaert, Lynn, Emanuel A. Fronhofer, Sébastien Lion, Christophe
Eizaguirre, Dries Bonte, Martijn Egas, Andrew P. Hendry, et al. 2019.
{``{Eco-evolutionary feedbacks---Theoretical models and
perspectives}.''} \emph{Functional Ecology} 33 (1): 13--30.
\url{https://doi.org/10.1111/1365-2435.13241}.

\leavevmode\vadjust pre{\hypertarget{ref-Kerr2009}{}}%
Kerr, Benjamin, and Peter Godfrey-Smith. 2009. {``{Generalization of the
price equation for evolutionary change}.''} \emph{Evolution} 63 (2):
531--36. \url{https://doi.org/10.1111/j.1558-5646.2008.00570.x}.

\leavevmode\vadjust pre{\hypertarget{ref-Kitcher1993}{}}%
Kitcher, Philip. 1993. \emph{{The advancement of science}}. New York:
Oxford University Press.

\leavevmode\vadjust pre{\hypertarget{ref-Lande1982}{}}%
Lande, Russell. 1982. {``{A quantitative genetic theory of life history
evolution}.''} \emph{Ecology} 63 (3): 607--15.
\url{https://doi.org/10.2307/1936778}.

\leavevmode\vadjust pre{\hypertarget{ref-Lehtonen2016}{}}%
Lehtonen, Jussi. 2016. {``{Multilevel selection in kin selection
language}.''} \emph{Trends in Ecology and Evolution} xx: 1--11.
\url{https://doi.org/10.1016/j.tree.2016.07.006}.

\leavevmode\vadjust pre{\hypertarget{ref-Lehtonen2018}{}}%
---------. 2018. {``{The Price equation, gradient dynamics, and
continuous trait game theory}.''} \emph{American Naturalist} 191 (1):
146--53. \url{https://doi.org/10.1086/694891}.

\leavevmode\vadjust pre{\hypertarget{ref-Lehtonen2020}{}}%
Lehtonen, Jussi, Samir Okasha, and Heikki Helanterä. 2020. {``{Fifty
years of the Price equation}.''} \emph{Philosophical Transactions of the
Royal Society B: Biological Sciences} 375 (1797): 20190350.
\url{https://doi.org/10.1098/rstb.2019.0350}.

\leavevmode\vadjust pre{\hypertarget{ref-Lewens2015}{}}%
Lewens, Tim. 2015. \emph{Cultural Evolution: Conceptual Challenges}.
Oxford University Press.
\url{https://doi.org/10.1093/acprof:oso/9780199674183.001.0001}.

\leavevmode\vadjust pre{\hypertarget{ref-Lion2018}{}}%
Lion, Sébastien. 2018. {``{Theoretical approaches in evolutionary
ecology: environmental feedback as a unifying perspective}.''}
\emph{American Naturalist} 191 (1).
\url{https://doi.org/10.1086/694865}.

\leavevmode\vadjust pre{\hypertarget{ref-Loreau2001}{}}%
Loreau, Michel, and Andy Hector. 2001. {``{Partitioning selection and
complementarity in biodiversity experiments}.''} \emph{Nature} 412:
72--76. \url{https://doi.org/10.1038/35083573}.

\leavevmode\vadjust pre{\hypertarget{ref-Luque2016}{}}%
Luque, Victor J. 2017. {``{One equation to rule them all: a
philosophical analysis of the Price equation}.''} \emph{Biology and
Philosophy} 32 (1): 1--29.
\url{https://doi.org/10.1007/s10539-016-9538-y}.

\leavevmode\vadjust pre{\hypertarget{ref-Luque2021}{}}%
Luque, Victor J., and Lorenzo Baravalle. 2021. {``{The mirror of
physics: on how the price equation can unify evolutionary biology}.''}
\emph{Synthese} 199: 12439--62.
\url{https://doi.org/10.1007/s11229-021-03339-6}.

\leavevmode\vadjust pre{\hypertarget{ref-MacCallum2012}{}}%
MacCallum, Robert M., Matthias Mauch, Austin Burt, and Armand M. Leroi.
2012. {``{Evolution of music by public choice}.''} \emph{Proceedings of
the National Academy of Sciences} 109 (30): 12081--86.
\url{https://doi.org/10.5061/dryad.h0228}.

\leavevmode\vadjust pre{\hypertarget{ref-MacColl2011}{}}%
MacColl, Andrew D. C. 2011. {``{The ecological causes of evolution}.''}
\emph{Trends in Ecology and Evolution} 26 (10): 514--22.
\url{https://doi.org/10.1016/j.tree.2011.06.009}.

\leavevmode\vadjust pre{\hypertarget{ref-Mateo-Tomas2017}{}}%
Mateo-Tomás, Patricia, Pedro P. Olea, Marcos Moleón, Nuria Selva, and
José A. Sánchez-Zapata. 2017. {``{Both rare and common species support
ecosystem services in scavenger communities}.''} \emph{Global Ecology
and Biogeography} 26 (12): 1459--70.
\url{https://doi.org/10.1111/geb.12673}.

\leavevmode\vadjust pre{\hypertarget{ref-Mayr1960}{}}%
Mayr, Ernst. 1959. {``Where Are We? Genetics and Twentieth Century
Darwinism.''} In \emph{Cold Spring Harbor Symposia on Quantitative
Biology}, 24:1--14. \url{https://doi.org/10.1101/SQB.1959.024.01.003}.

\leavevmode\vadjust pre{\hypertarget{ref-Morrison2000}{}}%
Morrison, Margaret. 2000. \emph{Unifying Scientific Theories: Physical
Concepts and Mathematical Structures}. Cambridge University Press.

\leavevmode\vadjust pre{\hypertarget{ref-Nagel1961}{}}%
Nagel, Ernest. 1961. \emph{The Structure of Science: Problems in the
Logic of Scientific Explanation}. New York, NY, USA: Harcourt, Brace \&
World.

\leavevmode\vadjust pre{\hypertarget{ref-Odling-smee2003}{}}%
Odling-Smee, F. John, Kevin N. Laland, and Marcus W. Feldman. 2003.
\emph{Niche Construction: The Neglected Process in Evolution}. Princeton
University Press.

\leavevmode\vadjust pre{\hypertarget{ref-Okasha2006}{}}%
Okasha, Samir. 2006. \emph{Evolution and the Levels of Selection}.
Oxford University Press.
\url{https://doi.org/10.1093/acprof:oso/9780199267972.001.0001}.

\leavevmode\vadjust pre{\hypertarget{ref-Okasha2020}{}}%
Okasha, Samir, and Jun Otsuka. 2020. {``{The Price equation and the
causal analysis of evolutionary change}.''} \emph{Philosophical
Transactions of the Royal Society B: Biological Sciences} 375 (1797):
20190365. \url{https://doi.org/10.1098/rstb.2019.0365}.

\leavevmode\vadjust pre{\hypertarget{ref-Page2002}{}}%
Page, Karen M., and Martin A. Nowak. 2002. {``Unifying Evolutionary
Dynamics.''} \emph{Journal of Theoretical Biology} 219 (1): 93--98.
\url{https://doi.org/10.1006/jtbi.2002.3112}.

\leavevmode\vadjust pre{\hypertarget{ref-Patel2018}{}}%
Patel, Swati, Michael H. Cortez, and Sebastian J. Schreiber. 2018.
{``{Partitioning the effects of eco-evolutionary feedbacks on community
stability}.''} \emph{American Naturalist} 191 (3): 1--29.
\url{https://doi.org/10.1101/104505}.

\leavevmode\vadjust pre{\hypertarget{ref-Pigliucci2009}{}}%
Pigliucci, Massimo. 2009. {``{An extended synthesis for evolutionary
biology}.''} \emph{Annals of the New York Academy of Sciences} 1168:
218--28. \url{https://doi.org/10.1111/j.1749-6632.2009.04578.x}.

\leavevmode\vadjust pre{\hypertarget{ref-Price1970}{}}%
Price, George R. 1970. {``{Selection and covariance}.''}
\url{https://doi.org/10.1038/227520a0}.

\leavevmode\vadjust pre{\hypertarget{ref-Price1972}{}}%
---------. 1972. {``{Extension of covariance selection mathematics}.''}
\emph{Annals of Human Genetics} 35 (4): 485--90.
\url{https://doi.org/10.1111/j.1469-1809.1957.tb01874.x}.

\leavevmode\vadjust pre{\hypertarget{ref-Price1995}{}}%
---------. 1995. {``{The nature of selection}.''} \emph{Journal of
Theoretical Biology} 175 (3): 389--96.
\url{https://doi.org/10.1006/jtbi.1995.0149}.

\leavevmode\vadjust pre{\hypertarget{ref-Queller2017}{}}%
Queller, David C. 2017. {``{Fundamental theorems of evolution}.''}
\emph{American Naturalist} 189 (4): 000--000.
\url{https://doi.org/10.1086/690937}.

\leavevmode\vadjust pre{\hypertarget{ref-Rice2004}{}}%
Rice, Sean H. 2004. \emph{{Evolutionary theory: mathematical and
conceptual foundations}}. Sinauer Associates Incorporated.

\leavevmode\vadjust pre{\hypertarget{ref-Rice2020}{}}%
---------. 2020. {``{Universal rules for the interaction of selection
and transmission in evolution}.''} \emph{Philosophical Transactions of
the Royal Society B: Biological Sciences} 375 (1797).
\url{https://doi.org/10.1098/rstb.2019.0353}.

\leavevmode\vadjust pre{\hypertarget{ref-Rice2009}{}}%
Rice, Sean H., and Anthony Papadopoulos. 2009. {``{Evolution with
stochastic fitness and stochastic migration}.''} \emph{PLoS One} 4 (10).
\url{https://doi.org/10.1371/journal.pone.0007130}.

\leavevmode\vadjust pre{\hypertarget{ref-Robertson1966}{}}%
Robertson, Alan. 1966. {``{A mathematical model of the culling process
in dairy cattle}.''} \emph{Animal Science} 8 (1): 95--108.
\url{https://doi.org/10.1017/S0003356100037752}.

\leavevmode\vadjust pre{\hypertarget{ref-Roff2008}{}}%
Roff, Derek A. 2008. {``{Defining fitness in evolutionary models}.''}
\emph{Journal of Genetics} 87 (4): 339--48.
\url{https://doi.org/10.1007/s12041-008-0056-9}.

\leavevmode\vadjust pre{\hypertarget{ref-Shennan2011}{}}%
Shennan, Stephen. 2011. {``{Descent with modification and the
archaeological record}.''} \emph{Philosophical Transactions of the Royal
Society B: Biological Sciences} 366 (1567): 1070--79.
\url{https://doi.org/10.1098/rstb.2010.0380}.

\leavevmode\vadjust pre{\hypertarget{ref-Turchin2001}{}}%
Turchin, Peter. 2001. {``Does Population Ecology Have General Laws?''}
\emph{Oikos} 94 (1): 17--26.
\url{https://doi.org/10.1034/j.1600-0706.2001.11310.x}.

\leavevmode\vadjust pre{\hypertarget{ref-Ulrich2024}{}}%
Ulrich, Werner, Nicholas J. Gotelli, Giovanni Strona, and William
Godsoe. 2024. {``Reconsidering the Price Equation: Benchmarking the
Analytical Power of Additive Partitioning in Ecology.''}
\emph{Ecological Modelling} 491: 110695.
\url{https://doi.org/10.1016/j.ecolmodel.2024.110695}.

\leavevmode\vadjust pre{\hypertarget{ref-VanVeelen2005}{}}%
van Veelen, Matthijs. 2005. {``{On the use of the Price equation}.''}
\emph{Journal of Theoretical Biology} 237 (4): 412--26.
\url{https://doi.org/10.1016/j.jtbi.2005.04.026}.

\leavevmode\vadjust pre{\hypertarget{ref-VanVeelen2020}{}}%
---------. 2020. {``{The problem with the Price equation}.''}
\emph{Philosophical Transactions of the Royal Society B: Biological
Sciences} 375 (1797): 20190355.
\url{https://doi.org/10.1098/rstb.2019.0355}.

\leavevmode\vadjust pre{\hypertarget{ref-VanVeelen2012}{}}%
van Veelen, Matthijs, Julián García, Maurice W. Sabelis, and Martijn
Egas. 2012. {``{Group selection and inclusive fitness are not
equivalent; the Price equation vs. models and statistics}.''}
\emph{Journal of Theoretical Biology} 299 (April): 64--80.
\url{https://doi.org/10.1016/j.jtbi.2011.07.025}.

\leavevmode\vadjust pre{\hypertarget{ref-Winfree2015}{}}%
Winfree, Rachael, Jeremy W. Fox, Neal M. Williams, James R. Reilly, and
Daniel P. Cariveau. 2015. {``{Abundance of common species, not species
richness, drives delivery of a real-world ecosystem service}.''}
\emph{Ecology Letters} 18: 626--35.
\url{https://doi.org/10.1111/ele.12424}.

\end{CSLReferences}

\end{document}
