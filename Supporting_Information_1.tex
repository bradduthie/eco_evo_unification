% Options for packages loaded elsewhere
\PassOptionsToPackage{unicode}{hyperref}
\PassOptionsToPackage{hyphens}{url}
\PassOptionsToPackage{dvipsnames,svgnames,x11names}{xcolor}
%
\documentclass[
]{article}
\usepackage{amsmath,amssymb}
\usepackage{iftex}
\ifPDFTeX
  \usepackage[T1]{fontenc}
  \usepackage[utf8]{inputenc}
  \usepackage{textcomp} % provide euro and other symbols
\else % if luatex or xetex
  \usepackage{unicode-math} % this also loads fontspec
  \defaultfontfeatures{Scale=MatchLowercase}
  \defaultfontfeatures[\rmfamily]{Ligatures=TeX,Scale=1}
\fi
\usepackage{lmodern}
\ifPDFTeX\else
  % xetex/luatex font selection
\fi
% Use upquote if available, for straight quotes in verbatim environments
\IfFileExists{upquote.sty}{\usepackage{upquote}}{}
\IfFileExists{microtype.sty}{% use microtype if available
  \usepackage[]{microtype}
  \UseMicrotypeSet[protrusion]{basicmath} % disable protrusion for tt fonts
}{}
\makeatletter
\@ifundefined{KOMAClassName}{% if non-KOMA class
  \IfFileExists{parskip.sty}{%
    \usepackage{parskip}
  }{% else
    \setlength{\parindent}{0pt}
    \setlength{\parskip}{6pt plus 2pt minus 1pt}}
}{% if KOMA class
  \KOMAoptions{parskip=half}}
\makeatother
\usepackage{xcolor}
\usepackage[margin=1in]{geometry}
\usepackage{graphicx}
\makeatletter
\def\maxwidth{\ifdim\Gin@nat@width>\linewidth\linewidth\else\Gin@nat@width\fi}
\def\maxheight{\ifdim\Gin@nat@height>\textheight\textheight\else\Gin@nat@height\fi}
\makeatother
% Scale images if necessary, so that they will not overflow the page
% margins by default, and it is still possible to overwrite the defaults
% using explicit options in \includegraphics[width, height, ...]{}
\setkeys{Gin}{width=\maxwidth,height=\maxheight,keepaspectratio}
% Set default figure placement to htbp
\makeatletter
\def\fps@figure{htbp}
\makeatother
\setlength{\emergencystretch}{3em} % prevent overfull lines
\providecommand{\tightlist}{%
  \setlength{\itemsep}{0pt}\setlength{\parskip}{0pt}}
\setcounter{secnumdepth}{-\maxdimen} % remove section numbering
% definitions for citeproc citations
\NewDocumentCommand\citeproctext{}{}
\NewDocumentCommand\citeproc{mm}{%
  \begingroup\def\citeproctext{#2}\cite{#1}\endgroup}
\makeatletter
 % allow citations to break across lines
 \let\@cite@ofmt\@firstofone
 % avoid brackets around text for \cite:
 \def\@biblabel#1{}
 \def\@cite#1#2{{#1\if@tempswa , #2\fi}}
\makeatother
\newlength{\cslhangindent}
\setlength{\cslhangindent}{1.5em}
\newlength{\csllabelwidth}
\setlength{\csllabelwidth}{3em}
\newenvironment{CSLReferences}[2] % #1 hanging-indent, #2 entry-spacing
 {\begin{list}{}{%
  \setlength{\itemindent}{0pt}
  \setlength{\leftmargin}{0pt}
  \setlength{\parsep}{0pt}
  % turn on hanging indent if param 1 is 1
  \ifodd #1
   \setlength{\leftmargin}{\cslhangindent}
   \setlength{\itemindent}{-1\cslhangindent}
  \fi
  % set entry spacing
  \setlength{\itemsep}{#2\baselineskip}}}
 {\end{list}}
\usepackage{calc}
\newcommand{\CSLBlock}[1]{\hfill\break\parbox[t]{\linewidth}{\strut\ignorespaces#1\strut}}
\newcommand{\CSLLeftMargin}[1]{\parbox[t]{\csllabelwidth}{\strut#1\strut}}
\newcommand{\CSLRightInline}[1]{\parbox[t]{\linewidth - \csllabelwidth}{\strut#1\strut}}
\newcommand{\CSLIndent}[1]{\hspace{\cslhangindent}#1}
\usepackage{fancyhdr}
\ifLuaTeX
  \usepackage{selnolig}  % disable illegal ligatures
\fi
\usepackage{bookmark}
\IfFileExists{xurl.sty}{\usepackage{xurl}}{} % add URL line breaks if available
\urlstyle{same}
\hypersetup{
  pdftitle={Foundations of community ecology: Supporting Information 1},
  colorlinks=true,
  linkcolor={blue},
  filecolor={Maroon},
  citecolor={Blue},
  urlcolor={Blue},
  pdfcreator={LaTeX via pandoc}}

\title{Foundations of community ecology: Supporting Information 1}
\author{A. Bradley Duthie\(^{1,a,*}\) and Victor J. Luque\(^{2,a}\)}
\date{{[}1{]} Department of Biological and Environmental Sciences,
University of Stirling, Stirling, Scotland {[}2{]} Department of
Philosophy, University of Valencia, Valencia, Spain {[}*{]}
Corresponding author:
\href{mailto:alexander.duthie@stir.ac.uk}{\nolinkurl{alexander.duthie@stir.ac.uk}}
{[}a{]} Equal contribution}

\begin{document}
\maketitle

This supporting information demonstrates how to derive well-established
models in population ecology and evolutionary biology from equation 1 in
the main text,

\[\Omega = \sum_{i=1}^{N} \left(\beta_{i} - \delta_{i} + 1 \right)\left(z_{i} + \Delta z_{i} \right).
\tag{1}
\]

In the main text, we derived both the Price equation and the birth-death
model from the above. Here we integrate interactions between individuals
to recover density-dependent population growth, and we integrate groups
within the population to recover multi-level selection. Finally, we
integrate both to model a system in which multi-level selection and
density dependent population change occur simultaneously. We can do this
in a way that clarifies model assumptions by allowing an individual
\(j\) to modulate the birth or death of the focal individual \(i\).

\section{Density-dependent population
growth}\label{density-dependent-population-growth}

There are two potential ways to model the incorporation of density
dependence into population growth. We start with what is likely the most
familiar model focusing on individual growth rate \(r_{i}\), then use a
slightly different model focusing on fitness \(w_{i}\). First note that
here we set \(\Omega = N_{t+1}\), and \(z_{i} = 1\) and
\(\Delta z_{i} = 0\) for all individuals as in the main text. We can
define \(r_{i} = \beta_{i} - \delta_{i}\) as an individual growth rate
for \(i\) (\citeproc{ref-Lion2018}{Lion 2018};
\citeproc{ref-Lion2023}{Lion, Sasaki, and Boots 2023}). In this case,

\[N_{t+1} = \sum_{i=1}^{N_{t}}\left(r_{i} + 1\right)
\tag{S1}
\]

Mathematically, the most general approach here would be to define
individual growth as a function of the entire system \(\mathbf{E}\),
\(r_{i}(\mathbf{E})\), where \(\mathbf{E}\) is a vector with elements
including any parameters potentially relevant to \(r_{i}\). Taking this
approach would recover a version of eqn 2 in Lion
(\citeproc{ref-Lion2018}{2018}) and permit any relationship between the
system and a focal individual's growth. Limiting our focus to the
effects of other individuals (\(j\)) and assuming that the effects of
these individuals are additive, let \(a_{ij|\cdot}\) be the effect of
individual \(j\) on the growth rate attributable to \(i\) conditioned on
all other individuals within the population such that
\(r_{i}\left(1 + \sum_{i = j}^{N}a_{ij|\cdot} \right)\) defines the
realised growth rate of \(i\),

\[N_{t+1} = \sum_{i=1}^{N_{t}}\left(r_{i}\left(1 - \sum_{i = j}^{N_{t}}a_{ij|\cdot} \right) + 1\right).
\tag{S2}
\]

Assuming that individual effects of \(j\) on \(i\) are also independent,
we can remove the condition,

\[N_{t+1} = \sum_{i=1}^{N_{t}}\left(r_{i}\left(1 - \sum_{i = j}^{N_{t}}a_{ij} \right) + 1\right).\]

Further assuming that all individuals have the same per capita effect
such that \(a = a_{ij}\) for any \(i\) and \(j\) pair (as might be
reasonable given resource competition in a well-mixed population),

\[N_{t+1} = \sum_{i=1}^{N_{t}}\left(r_{i}\left(1 - a N_{t} \right) + 1\right).\]

If \(r_{i}\) values are identical,

\[N_{t+1} = N_{t} + r N_{t}\left(1 - a N_{t} \right).
\tag{S3}
\]

Equation S3 therefore recovers a classic version of a discrete time
logistic growth by making assumptions from an exact model of
eco-evolutionary change.

An alternative approach would be to define model the effects of an
individual \(j\) on the fitness of \(i\) (\(w_{i}\)), thereby replacing
eqn S1 with \(N_{t+1} = \sum_{i=1}^{N_{t}}w_{i}\) and replacing eqn S2
with,

\[N_{t+1} = \sum_{i=1}^{N_{t}}w_{i}\left(1 - \sum_{i = j}^{N_{t}}\alpha_{ij|\cdot} \right).
\]

Note that we have used \(\alpha_{ij}\) to represent the effect of \(j\)
on the fitness of \(i\) for clarity in the sections below. By making the
same assumptions of additivity, independence, and identical effects such
that \(\alpha = \alpha_{ij}\) for all \(j\) on \(i\), and assuming
fitness is equal (\(w_{i} = \bar{w}\)), we can derive,

\[N_{t+1} = \bar{w}N_{t}(1 - \alpha N_{t}).
\tag{S4}
\]

This is an alternative way to express logistic growth.

\section{Multi-level selection}\label{multi-level-selection}

We can recover multi-level selection from our eqn 1. Here we derive the
original form of the multi-level Price (\citeproc{ref-Price1972}{1972})
equation as it appears in eqn 3.1 of Lehtonen
(\citeproc{ref-Lehtonen2020a}{2020}). Individuals belong to one of \(K\)
total groups where \(j\) indexes \(K\) groups and \(i\) indexes
individuals. Individuals do not overlap in group membership. The size of
group \(j\) is denoted as \(N_{j}\). Equation S5 below uses summations
to partition how individuals within each group contribute to \(\Omega\),

\[\Omega = \sum_{j=1}^{K}\sum_{i=1}^{N_{j}}\left(\beta_{j,i} - \delta_{j,i} + 1 \right)\left(z_{j, i} + \Delta z_{j, i}\right).
\tag{S5}
\]

In S5, indices \(\beta_{j, i}\), \(\delta_{j, i}\), and \(z_{j, i}\)
identify individual \(i\) in group \(j\). We set
\(w_{j, i} = \beta_{j, i} - \delta_{j, i} + 1\), and for simplicity let
\(\Delta z_{j, i} = 0\) (i.e., no transmission bias),

\[\Omega = \sum_{j=1}^{K}\sum_{i=1}^{N_{j}}w_{j,i}z_{j, i}.\]

For ease of presentation, with no loss of generality, we assume all
group sizes are equal with a group size of \(N_{j} = n\) for all \(j\).
If group sizes differ, then weighted expectations and covariances are
instead needed (\citeproc{ref-Lehtonen2020a}{Lehtonen 2020}). Given
equal group sizes, the total number of individuals (\(N\)) equals
\(K \times n\), and,

\[\frac{\Omega}{K n} = \left(\frac{1}{K}\right)\left(\frac{1}{n}\right)\sum_{j=1}^{K}\sum_{i=1}^{n}w_{j,i}z_{j, i}.\]

Rearranging,

\[\frac{\Omega}{K n} = \frac{1}{K}\sum_{j=1}^{K}\frac{1}{n}\sum_{i=1}^{n}w_{j,i}z_{j, i}.\]

The inner summation can be rewritten as an expectation for group \(j\),

\[\frac{\Omega}{K n} = \frac{1}{K}\sum_{j=1}^{K}\mathrm{E}_{j}\left(w_{j,i} z_{j,i}\right).\]

As in the main text, we note
\(\mathrm{E}(XY) = \mathrm{Cov}(X,Y) + \mathrm{E}(X)\mathrm{E}(Y)\).
Defining \(Cov_{j}(w_{j}, z_{j})\) as the covariance between \(w_{j,i}\)
and \(z_{j, i}\) for group \(j\),

\[\frac{\Omega}{K n} = \frac{1}{K}\sum_{j=1}^{K}\mathrm{Cov}_{j}\left(w_{j,i}, z_{j,i}\right) + \mathrm{E}_{j}\left(w_{j,i}\right)\mathrm{E}_{j}\left(z_{j,i}\right).\]

We can separate the summation for each term,

\[\frac{\Omega}{K n} = \frac{1}{K}\sum_{j=1}^{K}\mathrm{Cov}_{j}\left(w_{j,i}, z_{j,i}\right) + \frac{1}{K}\sum_{j=1}^{K}\mathrm{E}_{j}\left(w_{j,i}\right)\mathrm{E}_{j}\left(z_{j,i}\right).\]

Using the notation \(\bar{w}_{j} = E_{j}(w_{j,i})\) and
\(\bar{z}_{j} = E_{j}(z_{j,i})\) to indicate the expectation in group
\(j\),

\[\frac{\Omega}{K n} = \frac{1}{K}\sum_{j=1}^{K}\mathrm{Cov}_{j}\left(w_{j,i}, z_{j,i}\right) + \mathrm{E}\left(\bar{w}_{j} \bar{z}_{j} \right).\]

We can also rewrite the first term on the right-hand side as an
expectation,

\[\frac{\Omega}{K n} = \mathrm{E}\left(\mathrm{Cov}_{j}\left(w_{j}, z_{j}\right)\right) + \mathrm{E}\left(\bar{w}_{j} \bar{z}_{j} \right).\]

We can rearrange the second term on the right-hand side
(\(\bar{\bar{w}}\) indicates grand mean over all groups),

\[\frac{\Omega}{K n} = \mathrm{E}\left(\mathrm{Cov}_{j}\left(w_{j}, z_{j}\right)\right) +  \mathrm{Cov}\left(\bar{w}_{j}, \bar{z}_{j} \right) + \bar{\bar{w}}\bar{\bar{z}}.\]

As in the main text, note that \(Kn\bar{\bar{w}}\) accounts for
differences in total population size from \(t\) to \(t+1\), with
\(\bar{\bar{w}}\) being mean fitness across all groups. We can therefore
set \(\Omega = Kn\bar{\bar{w}}\bar{\bar{z'}}\), so,

\[\frac{Kn\bar{\bar{w}}\bar{\bar{z'}}}{K n} - \bar{\bar{w}}\bar{\bar{z}} = \mathrm{E}\left(\mathrm{Cov}_{j}\left(w_{j}, z_{j}\right)\right) +  \mathrm{Cov}\left(\bar{w}_{j}, \bar{z}_{j} \right).\]

Because \(\Delta \bar{\bar{z}} = \bar{\bar{z'}} - \bar{\bar{z}}\),

\[\bar{\bar{w}}\Delta \bar{\bar{z}} = \mathrm{Cov}\left(\bar{w}_{j}, \bar{z}_{j} \right) + \mathrm{E}\left(\mathrm{Cov}_{j}\left(w_{j}, z_{j}\right)\right).
\tag{S6}
\]

This recovers the multi-level Price (\citeproc{ref-Price1972}{1972})
equation (\citeproc{ref-Lehtonen2020a}{Lehtonen 2020}) from a starting
point of eco-evolutionary change in different groups. Equation S6 can be
found in Lehtonen (\citeproc{ref-Lehtonen2016}{2016}) B2.I, who then
derives a multi-level selection version of Hamilton's rule predicting
the evolution of altruism.

\section{Integration of ecology and
evolution}\label{integration-of-ecology-and-evolution}

For simplicity, we now focus on showing an integration between ecology
and evolution using a population with no multi-level selection and let
\(\Delta z_{j, i} = 0\) (i.e., no transmission bias). As above in the
section on density-dependent population growth, we define
\(w_{i} = \beta_{i} - \delta_{i} + 1\) and use \(\alpha_{i,j}\) to
represent the effect of \(j\) on the fitness of \(i\). Our starting
equation is therefore,

\[\Omega = \sum_{i=1}^{N}w_{i}\left(1 - \sum_{j=1}^{N}\alpha_{i, j}\right)z_{i}.
\tag{S7}
\]

We have already demonstrated that if we assume all individuals have the
same effect on a focal individual such that \(\alpha = \alpha_{i,j}\)
for all \(i\) and \(j\) pairs, we can recover equation S4 when
\(z_{i} = 1\) and \(\Omega\) is therefore interpreted as the count of
entities,

\[N_{t+1} = \bar{w} N_{t} \left(1 - \alpha N_{t}\right).\]

We now start from S7 to derive \(\Delta \bar{z}\). The objective is to
use our definition of eco-evolutionary change to simultaneously recover
how interactions between individuals affect population change and
evolutionary change.

We start by dividing both sides of S7 by \(N\),

\[\frac{\Omega}{N} = \frac{1}{N}\sum_{i=1}^{N}w_{i}\left(1 - \sum_{j=1}^{N}\alpha_{i, j}\right)z_{i}.
\tag{S8}
\]

We can express the right-hand side of eqn S8 as an expectation,

\[\frac{\Omega}{N} = \mathrm{E}\left(w_{i}\left(1 - \sum_{j=1}^{N}\alpha_{i, j}\right)z_{i}\right).
\tag{S9}
\]

We can rewrite the right-hand side in terms of covariances,

\[\frac{\Omega}{N} = \mathrm{Cov}\left(w_{i}\left(1 - \sum_{j=1}^{N}\alpha_{i, j}\right), z_{i}  \right) + \mathrm{E}\left(w_{i}\left(1 - \sum_{j=1}^{N}\alpha_{i, j}\right)\right)\mathrm{E}\left(z_{i}\right). 
\tag{S9}
\]

The expectations in the second term on the right-hand side can be
replaced with overbars to represent the mean,

\[\frac{\Omega}{N} = \mathrm{Cov}\left(w_{i}\left(1 - \sum_{j=1}^{N}\alpha_{i, j}\right), z_{i}  \right) + \overline{w_{i}\left(1 - \sum_{j=1}^{N}\alpha_{i, j}\right)}    \bar{z}_{i}. 
\tag{S9}
\]

In the main text, we noted that \(\Omega = N\bar{w}\bar{z}'\). Here mean
fitness incorporates individual interactions, therefore,

\[\Omega = N\overline{w_{i}\left(1 - \sum_{j=1}^{N}\alpha_{i, j}\right)}\bar{z}'.
\tag{S10}
\]

We can therefore rewrite S9,

\[\overline{w_{i}\left(1 - \sum_{j=1}^{N}\alpha_{i, j}\right)}\bar{z}' -  \overline{w_{i}\left(1 - \sum_{j=1}^{N}\alpha_{i, j}\right)}    \bar{z}_{i} = \mathrm{Cov}\left(w_{i}\left(1 - \sum_{j=1}^{N}\alpha_{i, j}\right), z_{i}  \right). 
\tag{S11}
\]

Noting again \(\Delta \bar{z} = \bar{z}' - \bar{z}\),

\[\overline{w_{i}\left(1 - \sum_{j=1}^{N}\alpha_{i, j}\right)}\Delta\bar{z} = \mathrm{Cov}\left(w_{i}\left(1 - \sum_{j=1}^{N}\alpha_{i, j}\right), z_{i}  \right). 
\tag{S12}
\]

We can rewrite the left-hand side of S12,

\[\overline{w_{i}\left(1 - \sum_{j=1}^{N}\alpha_{i, j}\right)}\Delta\bar{z} = \mathrm{Cov}\left(w_{i} - w_{i}\sum_{j=1}^{N}\alpha_{i, j}, z_{i}  \right). 
\tag{S13}
\]

The covariance term can be split without any additional assumptions,

\[\overline{w_{i}\left(1 - \sum_{j=1}^{N}\alpha_{i, j}\right)}\Delta\bar{z} = \mathrm{Cov}\left(w_{i}, z_{i}  \right) - \mathrm{Cov}\left(w_{i}\sum_{j=1}^{N}\alpha_{i, j}, z_{i}  \right). 
\tag{S14}
\]

If we are able to further assume that \(w_{i}\) and the summation over
\(\alpha_{i,j}\) are independent, then we could rewrite,

\[\overline{w_{i}\left(1 - \sum_{j=1}^{N}\alpha_{i, j}\right)}\Delta\bar{z} = \mathrm{Cov}\left(w_{i}, z_{i}  \right) - \mathrm{Cov}\left(\sum_{j=1}^{N}\alpha_{i, j}, z_{i}  \right)\bar{w_{i}}. 
\tag{S15}
\]

Partitioning fitness into different components with the Price equation
is commonplace. But this derivation highlights, e.g., the ecological and
evolutionary relationship between nonsocial and social components of
fitness, and population size. For example, the second term on the
right-hand side of S13 shows the covariance between the sum of social
interactions and a trait. When traits covary with the interaction
between sociality and fitness, they will have a stronger effect on trait
change. The magnitude of this second term will also increase with \(N\),
which reflects the stronger effect of the interaction between sociality
and fitness when there are more individuals interacting with a focal
individual.

\section*{References}\label{references}
\addcontentsline{toc}{section}{References}

\phantomsection\label{refs}
\begin{CSLReferences}{1}{0}
\bibitem[\citeproctext]{ref-Lehtonen2016}
Lehtonen, Jussi. 2016. {``{Multilevel selection in kin selection
language}.''} \emph{Trends in Ecology and Evolution} xx: 1--11.
\url{https://doi.org/10.1016/j.tree.2016.07.006}.

\bibitem[\citeproctext]{ref-Lehtonen2020a}
---------. 2020. {``The Price Equation and the Unity of Social Evolution
Theory.''} \emph{Philosophical Transactions of the Royal Society B:
Biological Sciences} 375: 20190362.
\url{https://doi.org/10.1098/rstb.2019.0362}.

\bibitem[\citeproctext]{ref-Lion2018}
Lion, Sébastien. 2018. {``{Theoretical approaches in evolutionary
ecology: environmental feedback as a unifying perspective}.''}
\emph{American Naturalist} 191 (1).
\url{https://doi.org/10.1086/694865}.

\bibitem[\citeproctext]{ref-Lion2023}
Lion, Sébastien, Akira Sasaki, and Mike Boots. 2023. {``Extending
Eco-Evolutionary Theory with Oligomorphic Dynamics.''} \emph{Ecology
Letters} 26 (September): S22--46.
\url{https://doi.org/10.1111/ele.14183}.

\bibitem[\citeproctext]{ref-Price1972}
Price, George R. 1972. {``{Extension of covariance selection
mathematics}.''} \emph{Annals of Human Genetics} 35 (4): 485--90.
\url{https://doi.org/10.1111/j.1469-1809.1957.tb01874.x}.

\end{CSLReferences}

\end{document}
